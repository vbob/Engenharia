\documentclass[paper=a4, fontsize=11pt]{scrartcl} % A4 paper and 11pt font size
    \usepackage{geometry}
    \geometry{
    a4paper,
    total={170mm,257mm},
    left=20mm,
    top=20mm,
    }

    \usepackage[T1]{fontenc} % Use 8-bit encoding that has 256 glyphs
    \usepackage{fourier} % Use the Adobe Utopia font for the document - comment this line to return to the LaTeX default
    \usepackage[brazil]{babel} % English language/hyphenation
	\usepackage{amsmath,amsfonts,amsthm} % Math packages
	\usepackage{multirow}
    \usepackage{systeme}
    \usepackage{enumerate}
    \usepackage{tikz}
	\usetikzlibrary{graphs}
	
	\usepackage{trimclip}
	\usepackage{adjustbox}
	\usepackage{graphicx}

    \usepackage{sectsty} % Allows customizing section commands
    \allsectionsfont{\normalfont\scshape} % Make all sections centered, the default font and small caps

    \setlength\parindent{0pt} % Removes all indentation from paragraphs - comment this line for an assignment with lots of text

    %----------------------------------------------------------------------------------------
    %	TITLE SECTION
    %----------------------------------------------------------------------------------------
    
    \newcommand{\horrule}[1]{\rule{\linewidth}{#1}} % Create horizontal rule command with 1 argument of height
    
    \title{	
    \normalfont \normalsize 
    \textsc{Instituto Federal de Educação, Ciência e Tecnologia de Mato Grosso} \\ [25pt] % Your university, school and/or department name(s)
    \horrule{0.5pt} \\[0.4cm] % Thin top horizontal rule
    \huge Inteligência Artificial \\ % The assignment title
    \huge Árvore de Decisão
    \horrule{2pt} \\[0.5cm] % Thick bottom horizontal rule
    }
    
    \author{Vitor Bruno de Oliveira Barth} % Your name
    
    \date{\normalsize\today} % Today's date or a custom date
    
    \begin{document}
    
    \maketitle
    
    %----------------------------------------------------------------------------------------
    %	Exercício 1
    %----------------------------------------------------------------------------------------
    
	\section{Monte a Árvore de Decisão dada a base de dados abaixo:}

	\begin{center}
    \begin{adjustbox}{max width=\textwidth} 
    \begin{tabular}{|cccccccccccc|}
        \hline
        \multicolumn{1}{|c|}{\multirow{2}{*}{Exemplo}}    & \multicolumn{10}{|c|}{Atributos}                                                                                                                                                                                                                                                                                              & \multicolumn{1}{c|}{\multirow{2}{*}{\begin{tabular}[c]{@{}c@{}}Obejetivo\\ VaiEsperar?\end{tabular}}} \\ \cline{2-11}
        \multicolumn{1}{|c|}{}                            & \multicolumn{1}{|c|}{Aternativa}   & \multicolumn{1}{|c|}{Bar} & \multicolumn{1}{|c|}{FimDeSemana} & \multicolumn{1}{|c|}{Fome} & \multicolumn{1}{|c|}{Clientes} & \multicolumn{1}{|c|}{Preço}  & \multicolumn{1}{|c|}{Chuva} & \multicolumn{1}{|c|}{Reserva} & \multicolumn{1}{|c|}{Tipo}       & \multicolumn{1}{|c|}{TempoEspera} &                                                      \\ \hline

        \multicolumn{1}{|c|}{${x_1}$}                     & \multicolumn{1}{|c|}{Sim}          & \multicolumn{1}{|c|}{Não} & \multicolumn{1}{|c|}{Não}         & \multicolumn{1}{|c|}{Sim}  & \multicolumn{1}{|c|}{Alguns}   & \multicolumn{1}{|c|}{SSS}   & \multicolumn{1}{|c|}{Não}   & \multicolumn{1}{|c|}{Sim}     & \multicolumn{1}{|c|}{Francês}    & \multicolumn{1}{|c|}{0-10}        & \multicolumn{1}{|c|}{$y_1$ = Sim}                     \\ 
        \multicolumn{1}{|c|}{${x_2}$}                     & \multicolumn{1}{|c|}{Sim}          & \multicolumn{1}{|c|}{Não} & \multicolumn{1}{|c|}{Não}         & \multicolumn{1}{|c|}{Sim}  & \multicolumn{1}{|c|}{Cheio}    & \multicolumn{1}{|c|}{S}     & \multicolumn{1}{|c|}{Não}   & \multicolumn{1}{|c|}{Não}     & \multicolumn{1}{|c|}{Tailandês}  & \multicolumn{1}{|c|}{30-60}       & \multicolumn{1}{|c|}{$y_2$ = Não}                     \\ 
        \multicolumn{1}{|c|}{${x_3}$}                     & \multicolumn{1}{|c|}{Não}          & \multicolumn{1}{|c|}{Sim} & \multicolumn{1}{|c|}{Não}         & \multicolumn{1}{|c|}{Não}  & \multicolumn{1}{|c|}{Alguns}   & \multicolumn{1}{|c|}{S}     & \multicolumn{1}{|c|}{Não}   & \multicolumn{1}{|c|}{Não}     & \multicolumn{1}{|c|}{Hamburger}  & \multicolumn{1}{|c|}{0-10}        & \multicolumn{1}{|c|}{$y_3$ = Sim}                     \\ 
        \multicolumn{1}{|c|}{${x_4}$}                     & \multicolumn{1}{|c|}{Sim}          & \multicolumn{1}{|c|}{Não} & \multicolumn{1}{|c|}{Sim}         & \multicolumn{1}{|c|}{Sim}  & \multicolumn{1}{|c|}{Cheio}    & \multicolumn{1}{|c|}{S}     & \multicolumn{1}{|c|}{Sim}   & \multicolumn{1}{|c|}{Não}     & \multicolumn{1}{|c|}{Tailandês}  & \multicolumn{1}{|c|}{10-30}       & \multicolumn{1}{|c|}{$y_4$ = Sim}                     \\ 
        \multicolumn{1}{|c|}{${x_5}$}                     & \multicolumn{1}{|c|}{Sim}          & \multicolumn{1}{|c|}{Não} & \multicolumn{1}{|c|}{Sim}         & \multicolumn{1}{|c|}{Não}  & \multicolumn{1}{|c|}{Cheio}    & \multicolumn{1}{|c|}{SSS}   & \multicolumn{1}{|c|}{Não}   & \multicolumn{1}{|c|}{Sim}     & \multicolumn{1}{|c|}{Francês}    & \multicolumn{1}{|c|}{>60}         & \multicolumn{1}{|c|}{$y_5$ = Não}                     \\ 
        \multicolumn{1}{|c|}{${x_6}$}                     & \multicolumn{1}{|c|}{Não}          & \multicolumn{1}{|c|}{Sim} & \multicolumn{1}{|c|}{Não}         & \multicolumn{1}{|c|}{Sim}  & \multicolumn{1}{|c|}{Alguns}   & \multicolumn{1}{|c|}{SS}    & \multicolumn{1}{|c|}{Sim}   & \multicolumn{1}{|c|}{Sim}     & \multicolumn{1}{|c|}{Italiano}   & \multicolumn{1}{|c|}{0-10}        & \multicolumn{1}{|c|}{$y_6$ = Sim}                     \\ 
        \multicolumn{1}{|c|}{${x_7}$}                     & \multicolumn{1}{|c|}{Não}          & \multicolumn{1}{|c|}{Sim} & \multicolumn{1}{|c|}{Não}         & \multicolumn{1}{|c|}{Não}  & \multicolumn{1}{|c|}{Ninguém}  & \multicolumn{1}{|c|}{S}     & \multicolumn{1}{|c|}{Sim}   & \multicolumn{1}{|c|}{Não}     & \multicolumn{1}{|c|}{Hamburger}  & \multicolumn{1}{|c|}{0-10}        & \multicolumn{1}{|c|}{$y_7$ = Não}                     \\ 
        \multicolumn{1}{|c|}{${x_8}$}                     & \multicolumn{1}{|c|}{Não}          & \multicolumn{1}{|c|}{Não} & \multicolumn{1}{|c|}{Não}         & \multicolumn{1}{|c|}{Sim}  & \multicolumn{1}{|c|}{Alguns}   & \multicolumn{1}{|c|}{SS}    & \multicolumn{1}{|c|}{Sim}   & \multicolumn{1}{|c|}{Sim}     & \multicolumn{1}{|c|}{Tailandês}  & \multicolumn{1}{|c|}{0-10}        & \multicolumn{1}{|c|}{$y_8$ = Sim}                     \\ 
        \multicolumn{1}{|c|}{${x_9}$}                     & \multicolumn{1}{|c|}{Não}          & \multicolumn{1}{|c|}{Sim} & \multicolumn{1}{|c|}{Sim}         & \multicolumn{1}{|c|}{Não}  & \multicolumn{1}{|c|}{Cheio}    & \multicolumn{1}{|c|}{S}     & \multicolumn{1}{|c|}{Sim}   & \multicolumn{1}{|c|}{Não}     & \multicolumn{1}{|c|}{Hamburger}  & \multicolumn{1}{|c|}{>60}         & \multicolumn{1}{|c|}{$y_9$ = Não}                     \\ 
        \multicolumn{1}{|c|}{${x_{10}}$}                    & \multicolumn{1}{|c|}{Sim}          & \multicolumn{1}{|c|}{Sim} & \multicolumn{1}{|c|}{Sim}         & \multicolumn{1}{|c|}{Sim}  & \multicolumn{1}{|c|}{Cheio}    & \multicolumn{1}{|c|}{SSS}   & \multicolumn{1}{|c|}{Não}   & \multicolumn{1}{|c|}{Sim}     & \multicolumn{1}{|c|}{Italiano}   & \multicolumn{1}{|c|}{10-30}       & \multicolumn{1}{|c|}{$y_{10}$ = Não}                  \\ 
        \multicolumn{1}{|c|}{$x_{11}$}                    & \multicolumn{1}{|c|}{Não}          & \multicolumn{1}{|c|}{Não} & \multicolumn{1}{|c|}{Não}         & \multicolumn{1}{|c|}{Não}  & \multicolumn{1}{|c|}{Ninguém}  & \multicolumn{1}{|c|}{S}     & \multicolumn{1}{|c|}{Não}   & \multicolumn{1}{|c|}{Não}     & \multicolumn{1}{|c|}{Tailandês}  & \multicolumn{1}{|c|}{0-10}        & \multicolumn{1}{|c|}{$y_{11}$ = Não}                  \\  
        \multicolumn{1}{|c|}{$x_{12}$}                    & \multicolumn{1}{|c|}{Sim}          & \multicolumn{1}{|c|}{Sim} & \multicolumn{1}{|c|}{Sim}         & \multicolumn{1}{|c|}{Sim}  & \multicolumn{1}{|c|}{Cheio}    & \multicolumn{1}{|c|}{S}     & \multicolumn{1}{|c|}{Não}   & \multicolumn{1}{|c|}{Não}     & \multicolumn{1}{|c|}{Hamburger}  & \multicolumn{1}{|c|}{30-60}       & \multicolumn{1}{|c|}{$y_{12}$ = Sim}                  \\ \hline
    \end{tabular}
    \end{adjustbox}
\end{center}
    


	\begin{enumerate}[(a)]
        \item \textbf{Escolha do Primeiro Nó}
		\par \par $S = [6,6]$

\vskip 0.1in
\par $S_{alternativa^+} = [3,3]$   \qquad $S_{alternativa^-} = [3,3]$
\par $Ganho(S, S_{alternativa}) = 0$

\vskip 0.1in
\par $S_{bar^+} = [3,3]$  \qquad $S_{bar^-} = [3,3]$
\par $Ganho(S, S_{bar}) = 0$

\vskip 0.1in
\par $S_{fimSemana^+} = [2,3]$  \qquad $S_{fimSemana^-} = [4,3]$
\par $Ganho(S, S_{fimSemana}) = 0.02$

\vskip 0.1in
\par $S_{fome^+} = [5,2]$ \qquad   $S_{fome^-} = [1,4]$
\par $Ganho(S, S_{fome}) = 0.158$

\vskip 0.1in
\par $S_{clientes^{alg}} = [4,0]$  \qquad $S_{clientes^{cheio}} = [2,4]$ \qquad $S_{clientes^{nin}} = [0,2]$
\par $Ganho(S, S_{clientes}) = 0.540$

\vskip 0.2in
\par $S_{preco^{S}} = [3,4]$  \qquad $S_{preco^{SS}} = [2,0]$ \qquad $S_{preco^{SSS}} = [1,2]$
\par $Ganho(S, S_{preco}) = 0.195$

\vskip 0.1in
\par $S_{chuva^{+}} = [3,2]$  \qquad $S_{chuva^{-}} = [3,4]$
\par $Ganho(S, S_{chuva}) = 0.02$

\vskip 0.1in
\par $S_{reserva^{+}} = [3,2]$  \qquad $S_{reserva^{-}} = [3,4]$
\par $Ganho(S, S_{reserva}) = 0.02$

\vskip 0.1in
\par $S_{tipo^{Fra}} = [1,1]$  \qquad $S_{tipo^{Tai}} = [2,2]$ 
    \qquad $S_{tipo^{Ita}} = [1,1]$ \qquad $S_{tipo^{Ham}} = [2,2]$ 
\par $Ganho(S, S_{tipo}) = 0$

\vskip 0.1in
\par $S_{tespera^{0-10}} = [4,2]$  \qquad $S_{tespera^{10-30}} = [1,1]$ 
    \qquad $S_{tespera^{30-60}} = [1,1]$ \qquad $S_{tespera^{>60}} = [0,2]$ 
\par $Ganho(S, S_{tespera}) = 0.207$

\vskip 0.25in
\hfil
\begin{tikzpicture}[sibling distance=10em,
    every node/.style = {shape=rectangle, 
      draw, align=center,
      top color=green!20, bottom color=green!20}]]
    \node {Observa Quantidade \\ de Clientes}
        child { node {Ninguém} child { node {NÃO ESPERA}  } }
        child { node {Alguns} child { node {ESPERA} } }
        child { node {Cheio} };
  \end{tikzpicture}
		\vskip 0.3in
		\item \textbf{Escolha do Segundo Nó}
		\par 
\par $S_{cheio} = [2,4]$

\vskip 0.1in
\par $S_{alternativa^+} = [2,3]$   \qquad $S_{alternativa^-} = [0,1]$
\par $Ganho(S_{cheio}, S_{alternativa}) = 0.109$

\vskip 0.1in
\par $S_{bar^+} = [1,2]$  \qquad $S_{bar^-} = [1,2]$
\par $Ganho(S_{cheio}, S_{bar}) = 0$

\vskip 0.1in
\par $S_{fimSemana^+} = [2,3]$  \qquad $S_{fimSemana^-} = [0,1]$
\par $Ganho(S_{cheio}, S_{fimSemana}) = 0.109$

\vskip 0.1in
\par $S_{fome^+} = [2,2]$ \qquad   $S_{fome^-} = [0,2]$
\par $Ganho(S_{cheio}, S_{fome}) = 0.251$

\vskip 0.1in
\par $S_{preco^{S}} = [2,2]$ \qquad $S_{preco^{SSS}} = [0,2]$
\par $Ganho(S_{cheio}, S_{preco}) = 0.251$

\vskip 0.1in
\par $S_{chuva^{+}} = [1,1]$  \qquad $S_{chuva^{-}} = [1,3]$
\par $Ganho(S_{cheio}, S_{chuva}) = 0.044$

\vskip 0.1in
\par $S_{reserva^{+}} = [0,2]$  \qquad $S_{reserva^{-}} = [2,2]$
\par $Ganho(S_{cheio}, S_{reserva}) = 0.25$

\vskip 0.5in
\par $S_{tipo^{Fra}} = [0,1]$  \qquad $S_{tipo^{Tai}} = [1,1]$ 
    \qquad $S_{tipo^{Ita}} = [0,1]$ \qquad $S_{tipo^{Ham}} = [1,1]$ 
\par $Ganho(S_{cheio}, S_{tipo}) = 0.251$

\vskip 0.1in
\par $S_{tespera^{10-30}} = [1,1]$ \qquad $S_{tespera^{30-60}} = [1,1]$ \qquad $S_{tespera^{>60}} = [0,2]$ 
\par $Ganho(S_{cheio_{fome}}, S_{tespera}) = 0.251$

\vskip 0.25in
\hfil
\begin{tikzpicture}[sibling distance=10em,
    every node/.style = {shape=rectangle, 
      draw, align=center,
      top color=green!20, bottom color=green!20}]]
    \node {Observa Quantidade \\ de Clientes}
        child { node {Ninguém} child { node {NÃO ESPERA}  } }
        child { node {Alguns} child { node {ESPERA}  } }
        child { node {Cheio} child { 
            node {Está com fome?} child { node {Sim}  } child { node {Não} child { node {NÃO ESPERA}  }  }
            }
        };
  \end{tikzpicture}
		\vskip 0.3in
		\item \textbf{Escolha do Terceiro Nó}
		\par 
\par $S_{cheio_{fome}} = [2,2]$

\vskip 0.3in
\par $S_{alternativa^{+}} = [1,1]$ \qquad $S_{alternativa^{-}} = [1,1]$
\par $H(S_{alternativa^{+}}) = -\frac{1}{2} \log_2 \frac{1}{2}- \frac{1}{2} \log_2 \frac{1}{2} = 1.000$
\par $H(S_{alternativa^{-}}) = -\frac{1}{2} \log_2 \frac{1}{2}- \frac{1}{2} \log_2 \frac{1}{2} = 1.000$
\par $Ganho(S, S_{alternativa}) = 1.000-\frac{2}{4} * 1.000-\frac{2}{4} * 1.000 = 0.000$

\vskip 0.3in
\par $S_{bar^{+}} = [1,1]$ \qquad $S_{bar^{-}} = [1,1]$
\par $H(S_{bar^{+}}) = -\frac{1}{2} \log_2 \frac{1}{2}- \frac{1}{2} \log_2 \frac{1}{2} = 1.000$
\par $H(S_{bar^{-}}) = -\frac{1}{2} \log_2 \frac{1}{2}- \frac{1}{2} \log_2 \frac{1}{2} = 1.000$
\par $Ganho(S, S_{bar}) = 1.000-\frac{2}{4} * 1.000-\frac{2}{4} * 1.000 = 0.000$

\vskip 0.3in
\par $S_{fimSemana^{+}} = [2,1]$ \qquad $S_{fimSemana^{-}} = [0,1]$
\par $H(S_{fimSemana^{+}}) = -\frac{2}{3} \log_2 \frac{2}{3}- \frac{1}{3} \log_2 \frac{1}{3} = 0.918$
\par $H(S_{fimSemana^{-}}) = -\frac{0}{1} \log_2 \frac{0}{1}- \frac{1}{1} \log_2 \frac{1}{1} = 0.000$
\par $Ganho(S, S_{fimSemana}) = 1.000-\frac{3}{4} * 0.918-\frac{1}{4} * 0.000 = 0.311$

\vskip 0.3in
\par $S_{preço^{\$}} = [1,0]$ \qquad $S_{preço^{\$\$\$}} = [2,1]$
\par $H(S_{preço^{\$}}) = -\frac{1}{1} \log_2 \frac{1}{1}- \frac{0}{1} \log_2 \frac{0}{1} = 0.000$
\par $H(S_{preço^{\$\$\$}}) = -\frac{2}{3} \log_2 \frac{2}{3}- \frac{1}{3} \log_2 \frac{1}{3} = 0.918$
\par $Ganho(S, S_{preço}) = 1.000-\frac{1}{4} * 0.000-\frac{3}{4} * 0.918 = 0.311$

\vskip 0.3in
\par $S_{chuva^{+}} = [1,0]$ \qquad $S_{chuva^{-}} = [1,2]$
\par $H(S_{chuva^{+}}) = -\frac{1}{1} \log_2 \frac{1}{1}- \frac{0}{1} \log_2 \frac{0}{1} = 0.000$
\par $H(S_{chuva^{-}}) = -\frac{1}{3} \log_2 \frac{1}{3}- \frac{2}{3} \log_2 \frac{2}{3} = 0.918$
\par $Ganho(S, S_{chuva}) = 1.000-\frac{1}{4} * 0.000-\frac{3}{4} * 0.918 = 0.311$

\vskip 0.3in
\par $S_{reserva^{+}} = [0,1]$ \qquad $S_{reserva^{-}} = [2,1]$
\par $H(S_{reserva^{+}}) = -\frac{0}{1} \log_2 \frac{0}{1}- \frac{1}{1} \log_2 \frac{1}{1} = 0.000$
\par $H(S_{reserva^{-}}) = -\frac{2}{3} \log_2 \frac{2}{3}- \frac{1}{3} \log_2 \frac{1}{3} = 0.918$
\par $Ganho(S, S_{reserva}) = 1.000-\frac{1}{4} * 0.000-\frac{3}{4} * 0.918 = 0.311$

\vskip 0.3in
\par $S_{tipo^{Tailandês}} = [1,1]$ \qquad $S_{tipo^{Italiano}} = [0,1]$\par $S_{tipo^{Hamburger}} = [1,0]$
\par $H(S_{tipo^{Tailandês}}) = -\frac{1}{2} \log_2 \frac{1}{2}- \frac{1}{2} \log_2 \frac{1}{2} = 1.000$
\par $H(S_{tipo^{Italiano}}) = -\frac{0}{1} \log_2 \frac{0}{1}- \frac{1}{1} \log_2 \frac{1}{1} = 0.000$
\par $H(S_{tipo^{Hamburger}}) = -\frac{1}{1} \log_2 \frac{1}{1}- \frac{0}{1} \log_2 \frac{0}{1} = 0.000$
\par $Ganho(S, S_{tipo}) = 1.000-\frac{2}{4} * 1.000-\frac{1}{4} * 0.000-\frac{1}{4} * 0.000 = 0.500$

\vskip 0.3in
\par $S_{tempoEspera^{10-30}} = [1,1]$ \qquad $S_{tempoEspera^{30-60}} = [1,1]$
\par $H(S_{tempoEspera^{10-30}}) = -\frac{1}{2} \log_2 \frac{1}{2}- \frac{1}{2} \log_2 \frac{1}{2} = 1.000$
\par $H(S_{tempoEspera^{30-60}}) = -\frac{1}{2} \log_2 \frac{1}{2}- \frac{1}{2} \log_2 \frac{1}{2} = 1.000$
\par $Ganho(S, S_{tempoEspera}) = 1.000-\frac{2}{4} * 1.000-\frac{2}{4} * 1.000 = 0.000$

\vskip 0.4in
\hfil
\begin{tikzpicture}[sibling distance=10em,
    every node/.style = {shape=rectangle, 
      draw, align=center,
      top color=green!20, bottom color=green!20}]]
    \node {Observa Quantidade \\ de Clientes}
        child { node {Ninguém} child { node {NÃO ESPERA}  } }
        child { node {Alguns} child { node {ESPERA}  } }
        child [missing]
        child { node {Cheio} child { 
            node {Está com fome?} 
            child { node {Sim} child { node {Qual o Tipo?} 
                 child { node {Tailandês} }   
                 child { node {Italiano}  child { node {NÃO ESPERA}  }  } 
                 child { node {Hamburguer} child { node {ESPERA}  } } 
            }   }
            child [missing]
            child { node {Não} child { node {NÃO ESPERA}  }  }
            }
        };
  \end{tikzpicture}
		\vskip 0.3in
		\item \textbf{Escolha do Quarto Nó}
        \par 
\par $S_{cheio_{fome_{tai}}} = [1,1]$

\vskip 0.31in
\par $S_{alternativa^{+}} = [1,1]$
\par $H(S_{alternativa^{+}}) = -\frac{1}{2} \log_2 \frac{1}{2}- \frac{1}{2} \log_2 \frac{1}{2} = 1.000$
\par $Ganho(S, S_{alternativa}) = 1.000-\frac{2}{2} * 1.000 = 0.000$

\vskip 0.3in
\par $S_{bar^{+}} = [1,1]$
\par $H(S_{bar^{+}}) = -\frac{1}{2} \log_2 \frac{1}{2}- \frac{1}{2} \log_2 \frac{1}{2} = 1.000$
\par $Ganho(S, S_{bar}) = 1.000-\frac{2}{2} * 1.000 = 0.000$

\vskip 0.3in
\par $S_{fimSemana^{+}} = [1,0]$ \qquad $S_{fimSemana^-} = [0,1]$
\par $H(S_{fimSemana^{+}}) = -\frac{1}{1} \log_2 \frac{1}{1}- \frac{0}{1} \log_2 \frac{0}{1} = 0.000$
\par $H(S_{fimSemana^-}) = -\frac{0}{1} \log_2 \frac{0}{1}- \frac{1}{1} \log_2 \frac{1}{1} = 0.000$
\par $Ganho(S, S_{fimSemana}) = 1.000-\frac{1}{2} * 0.000-\frac{1}{2} * 0.000 = 1.000$

\vskip 0.3in
\par $S_{preço^{\$}} = [1,1]$
\par $H(S_{preço^{\$}}) = -\frac{1}{2} \log_2 \frac{1}{2}- \frac{1}{2} \log_2 \frac{1}{2} = 1.000$
\par $Ganho(S, S_{preço}) = 1.000-\frac{2}{2} * 1.000 = 0.000$

\vskip 0.3in
\par $S_{chuva^+} = [1,0]$ \qquad $S_{chuva^-} = [0,1]$
\par $H(S_{chuva^+}) = -\frac{1}{1} \log_2 \frac{1}{1}- \frac{0}{1} \log_2 \frac{0}{1} = 0.000$
\par $H(S_{chuva^-}) = -\frac{0}{1} \log_2 \frac{0}{1}- \frac{1}{1} \log_2 \frac{1}{1} = 0.000$
\par $Ganho(S, S_{chuva}) = 1.000-\frac{1}{2} * 0.000-\frac{1}{2} * 0.000 = 1.000$

\vskip 0.3in
\par $S_{reserva^+} = [1,1]$
\par $H(S_{reserva^+}) = -\frac{1}{2} \log_2 \frac{1}{2}- \frac{1}{2} \log_2 \frac{1}{2} = 1.000$
\par $Ganho(S, S_{reserva}) = 1.000-\frac{2}{2} * 1.000 = 0.000$

\vskip 0.3in
\par $S_{tempoEspera^{10-30}} = [1,0]$ \qquad $S_{tempoEspera^{30-60}} = [0,1]$
\par $H(S_{tempoEspera^{10-30}}) = -\frac{1}{1} \log_2 \frac{1}{1}- \frac{0}{1} \log_2 \frac{0}{1} = 0.000$
\par $H(S_{tempoEspera^{30-60}}) = -\frac{0}{1} \log_2 \frac{0}{1}- \frac{1}{1} \log_2 \frac{1}{1} = 0.000$
\par $Ganho(S, S_{tempoEspera}) = 1.000-\frac{1}{2} * 0.000-\frac{1}{2} * 0.000 = 1.000$

\vskip 0.25in
\hfil
\begin{tikzpicture}[sibling distance=10em,
    every node/.style = {shape=rectangle, 
      draw, align=center,
      top color=green!20, bottom color=green!20}]]
    \node {Observa Quantidade \\ de Clientes}
        child { node {Ninguém} child { node {NÃO ESPERA}  } }
        child { node {Alguns} child { node {ESPERA}  } }
        child [missing]
        child { node {Cheio} child { 
            node {Está com fome?} 
            child { node {Sim} child { node {Qual o Tipo?} 
                 child { node {Tailandês} 
                    child { node {Está Chovendo?}  
                        child { node {Sim}  child { node {ESPERA}  } }
                        child { node {Não}  child { node {NÃO ESPERA}  } }
                    }  
                 }   
                 child [missing]
                 child { node {Italiano}  child { node {NÃO ESPERA}  }  } 
                 child { node {Hamburguer} child { node {ESPERA}  } } 
            }   }
            child [missing]
            child { node {Não} child { node {NÃO ESPERA}  }  }
            }
        };
  \end{tikzpicture}
    \end{enumerate}




    \end{document}