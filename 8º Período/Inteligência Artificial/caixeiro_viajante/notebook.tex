
% Default to the notebook output style

    


% Inherit from the specified cell style.




    
\documentclass[11pt]{article}

    
    
    \usepackage[T1]{fontenc}
    % Nicer default font (+ math font) than Computer Modern for most use cases
    \usepackage{mathpazo}

    % Basic figure setup, for now with no caption control since it's done
    % automatically by Pandoc (which extracts ![](path) syntax from Markdown).
    \usepackage{graphicx}
    % We will generate all images so they have a width \maxwidth. This means
    % that they will get their normal width if they fit onto the page, but
    % are scaled down if they would overflow the margins.
    \makeatletter
    \def\maxwidth{\ifdim\Gin@nat@width>\linewidth\linewidth
    \else\Gin@nat@width\fi}
    \makeatother
    \let\Oldincludegraphics\includegraphics
    % Set max figure width to be 80% of text width, for now hardcoded.
    \renewcommand{\includegraphics}[1]{\Oldincludegraphics[width=.8\maxwidth]{#1}}
    % Ensure that by default, figures have no caption (until we provide a
    % proper Figure object with a Caption API and a way to capture that
    % in the conversion process - todo).
    \usepackage{caption}
    \DeclareCaptionLabelFormat{nolabel}{}
    \captionsetup{labelformat=nolabel}

    \usepackage{adjustbox} % Used to constrain images to a maximum size 
    \usepackage{xcolor} % Allow colors to be defined
    \usepackage{enumerate} % Needed for markdown enumerations to work
    \usepackage{geometry} % Used to adjust the document margins
    \usepackage{amsmath} % Equations
    \usepackage{amssymb} % Equations
    \usepackage{textcomp} % defines textquotesingle
    % Hack from http://tex.stackexchange.com/a/47451/13684:
    \AtBeginDocument{%
        \def\PYZsq{\textquotesingle}% Upright quotes in Pygmentized code
    }
    \usepackage{upquote} % Upright quotes for verbatim code
    \usepackage{eurosym} % defines \euro
    \usepackage[mathletters]{ucs} % Extended unicode (utf-8) support
    \usepackage[utf8x]{inputenc} % Allow utf-8 characters in the tex document
    \usepackage{fancyvrb} % verbatim replacement that allows latex
    \usepackage{grffile} % extends the file name processing of package graphics 
                         % to support a larger range 
    % The hyperref package gives us a pdf with properly built
    % internal navigation ('pdf bookmarks' for the table of contents,
    % internal cross-reference links, web links for URLs, etc.)
    \usepackage{hyperref}
    \usepackage{longtable} % longtable support required by pandoc >1.10
    \usepackage{booktabs}  % table support for pandoc > 1.12.2
    \usepackage[inline]{enumitem} % IRkernel/repr support (it uses the enumerate* environment)
    \usepackage[normalem]{ulem} % ulem is needed to support strikethroughs (\sout)
                                % normalem makes italics be italics, not underlines
    

    
    
    % Colors for the hyperref package
    \definecolor{urlcolor}{rgb}{0,.145,.698}
    \definecolor{linkcolor}{rgb}{.71,0.21,0.01}
    \definecolor{citecolor}{rgb}{.12,.54,.11}

    % ANSI colors
    \definecolor{ansi-black}{HTML}{3E424D}
    \definecolor{ansi-black-intense}{HTML}{282C36}
    \definecolor{ansi-red}{HTML}{E75C58}
    \definecolor{ansi-red-intense}{HTML}{B22B31}
    \definecolor{ansi-green}{HTML}{00A250}
    \definecolor{ansi-green-intense}{HTML}{007427}
    \definecolor{ansi-yellow}{HTML}{DDB62B}
    \definecolor{ansi-yellow-intense}{HTML}{B27D12}
    \definecolor{ansi-blue}{HTML}{208FFB}
    \definecolor{ansi-blue-intense}{HTML}{0065CA}
    \definecolor{ansi-magenta}{HTML}{D160C4}
    \definecolor{ansi-magenta-intense}{HTML}{A03196}
    \definecolor{ansi-cyan}{HTML}{60C6C8}
    \definecolor{ansi-cyan-intense}{HTML}{258F8F}
    \definecolor{ansi-white}{HTML}{C5C1B4}
    \definecolor{ansi-white-intense}{HTML}{A1A6B2}

    % commands and environments needed by pandoc snippets
    % extracted from the output of `pandoc -s`
    \providecommand{\tightlist}{%
      \setlength{\itemsep}{0pt}\setlength{\parskip}{0pt}}
    \DefineVerbatimEnvironment{Highlighting}{Verbatim}{commandchars=\\\{\}}
    % Add ',fontsize=\small' for more characters per line
    \newenvironment{Shaded}{}{}
    \newcommand{\KeywordTok}[1]{\textcolor[rgb]{0.00,0.44,0.13}{\textbf{{#1}}}}
    \newcommand{\DataTypeTok}[1]{\textcolor[rgb]{0.56,0.13,0.00}{{#1}}}
    \newcommand{\DecValTok}[1]{\textcolor[rgb]{0.25,0.63,0.44}{{#1}}}
    \newcommand{\BaseNTok}[1]{\textcolor[rgb]{0.25,0.63,0.44}{{#1}}}
    \newcommand{\FloatTok}[1]{\textcolor[rgb]{0.25,0.63,0.44}{{#1}}}
    \newcommand{\CharTok}[1]{\textcolor[rgb]{0.25,0.44,0.63}{{#1}}}
    \newcommand{\StringTok}[1]{\textcolor[rgb]{0.25,0.44,0.63}{{#1}}}
    \newcommand{\CommentTok}[1]{\textcolor[rgb]{0.38,0.63,0.69}{\textit{{#1}}}}
    \newcommand{\OtherTok}[1]{\textcolor[rgb]{0.00,0.44,0.13}{{#1}}}
    \newcommand{\AlertTok}[1]{\textcolor[rgb]{1.00,0.00,0.00}{\textbf{{#1}}}}
    \newcommand{\FunctionTok}[1]{\textcolor[rgb]{0.02,0.16,0.49}{{#1}}}
    \newcommand{\RegionMarkerTok}[1]{{#1}}
    \newcommand{\ErrorTok}[1]{\textcolor[rgb]{1.00,0.00,0.00}{\textbf{{#1}}}}
    \newcommand{\NormalTok}[1]{{#1}}
    
    % Additional commands for more recent versions of Pandoc
    \newcommand{\ConstantTok}[1]{\textcolor[rgb]{0.53,0.00,0.00}{{#1}}}
    \newcommand{\SpecialCharTok}[1]{\textcolor[rgb]{0.25,0.44,0.63}{{#1}}}
    \newcommand{\VerbatimStringTok}[1]{\textcolor[rgb]{0.25,0.44,0.63}{{#1}}}
    \newcommand{\SpecialStringTok}[1]{\textcolor[rgb]{0.73,0.40,0.53}{{#1}}}
    \newcommand{\ImportTok}[1]{{#1}}
    \newcommand{\DocumentationTok}[1]{\textcolor[rgb]{0.73,0.13,0.13}{\textit{{#1}}}}
    \newcommand{\AnnotationTok}[1]{\textcolor[rgb]{0.38,0.63,0.69}{\textbf{\textit{{#1}}}}}
    \newcommand{\CommentVarTok}[1]{\textcolor[rgb]{0.38,0.63,0.69}{\textbf{\textit{{#1}}}}}
    \newcommand{\VariableTok}[1]{\textcolor[rgb]{0.10,0.09,0.49}{{#1}}}
    \newcommand{\ControlFlowTok}[1]{\textcolor[rgb]{0.00,0.44,0.13}{\textbf{{#1}}}}
    \newcommand{\OperatorTok}[1]{\textcolor[rgb]{0.40,0.40,0.40}{{#1}}}
    \newcommand{\BuiltInTok}[1]{{#1}}
    \newcommand{\ExtensionTok}[1]{{#1}}
    \newcommand{\PreprocessorTok}[1]{\textcolor[rgb]{0.74,0.48,0.00}{{#1}}}
    \newcommand{\AttributeTok}[1]{\textcolor[rgb]{0.49,0.56,0.16}{{#1}}}
    \newcommand{\InformationTok}[1]{\textcolor[rgb]{0.38,0.63,0.69}{\textbf{\textit{{#1}}}}}
    \newcommand{\WarningTok}[1]{\textcolor[rgb]{0.38,0.63,0.69}{\textbf{\textit{{#1}}}}}
    
    
    % Define a nice break command that doesn't care if a line doesn't already
    % exist.
    \def\br{\hspace*{\fill} \\* }
    % Math Jax compatability definitions
    \def\gt{>}
    \def\lt{<}
    % Document parameters
    \title{caixeiro\_viajante}
    
    
    

    % Pygments definitions
    
\makeatletter
\def\PY@reset{\let\PY@it=\relax \let\PY@bf=\relax%
    \let\PY@ul=\relax \let\PY@tc=\relax%
    \let\PY@bc=\relax \let\PY@ff=\relax}
\def\PY@tok#1{\csname PY@tok@#1\endcsname}
\def\PY@toks#1+{\ifx\relax#1\empty\else%
    \PY@tok{#1}\expandafter\PY@toks\fi}
\def\PY@do#1{\PY@bc{\PY@tc{\PY@ul{%
    \PY@it{\PY@bf{\PY@ff{#1}}}}}}}
\def\PY#1#2{\PY@reset\PY@toks#1+\relax+\PY@do{#2}}

\expandafter\def\csname PY@tok@w\endcsname{\def\PY@tc##1{\textcolor[rgb]{0.73,0.73,0.73}{##1}}}
\expandafter\def\csname PY@tok@c\endcsname{\let\PY@it=\textit\def\PY@tc##1{\textcolor[rgb]{0.25,0.50,0.50}{##1}}}
\expandafter\def\csname PY@tok@cp\endcsname{\def\PY@tc##1{\textcolor[rgb]{0.74,0.48,0.00}{##1}}}
\expandafter\def\csname PY@tok@k\endcsname{\let\PY@bf=\textbf\def\PY@tc##1{\textcolor[rgb]{0.00,0.50,0.00}{##1}}}
\expandafter\def\csname PY@tok@kp\endcsname{\def\PY@tc##1{\textcolor[rgb]{0.00,0.50,0.00}{##1}}}
\expandafter\def\csname PY@tok@kt\endcsname{\def\PY@tc##1{\textcolor[rgb]{0.69,0.00,0.25}{##1}}}
\expandafter\def\csname PY@tok@o\endcsname{\def\PY@tc##1{\textcolor[rgb]{0.40,0.40,0.40}{##1}}}
\expandafter\def\csname PY@tok@ow\endcsname{\let\PY@bf=\textbf\def\PY@tc##1{\textcolor[rgb]{0.67,0.13,1.00}{##1}}}
\expandafter\def\csname PY@tok@nb\endcsname{\def\PY@tc##1{\textcolor[rgb]{0.00,0.50,0.00}{##1}}}
\expandafter\def\csname PY@tok@nf\endcsname{\def\PY@tc##1{\textcolor[rgb]{0.00,0.00,1.00}{##1}}}
\expandafter\def\csname PY@tok@nc\endcsname{\let\PY@bf=\textbf\def\PY@tc##1{\textcolor[rgb]{0.00,0.00,1.00}{##1}}}
\expandafter\def\csname PY@tok@nn\endcsname{\let\PY@bf=\textbf\def\PY@tc##1{\textcolor[rgb]{0.00,0.00,1.00}{##1}}}
\expandafter\def\csname PY@tok@ne\endcsname{\let\PY@bf=\textbf\def\PY@tc##1{\textcolor[rgb]{0.82,0.25,0.23}{##1}}}
\expandafter\def\csname PY@tok@nv\endcsname{\def\PY@tc##1{\textcolor[rgb]{0.10,0.09,0.49}{##1}}}
\expandafter\def\csname PY@tok@no\endcsname{\def\PY@tc##1{\textcolor[rgb]{0.53,0.00,0.00}{##1}}}
\expandafter\def\csname PY@tok@nl\endcsname{\def\PY@tc##1{\textcolor[rgb]{0.63,0.63,0.00}{##1}}}
\expandafter\def\csname PY@tok@ni\endcsname{\let\PY@bf=\textbf\def\PY@tc##1{\textcolor[rgb]{0.60,0.60,0.60}{##1}}}
\expandafter\def\csname PY@tok@na\endcsname{\def\PY@tc##1{\textcolor[rgb]{0.49,0.56,0.16}{##1}}}
\expandafter\def\csname PY@tok@nt\endcsname{\let\PY@bf=\textbf\def\PY@tc##1{\textcolor[rgb]{0.00,0.50,0.00}{##1}}}
\expandafter\def\csname PY@tok@nd\endcsname{\def\PY@tc##1{\textcolor[rgb]{0.67,0.13,1.00}{##1}}}
\expandafter\def\csname PY@tok@s\endcsname{\def\PY@tc##1{\textcolor[rgb]{0.73,0.13,0.13}{##1}}}
\expandafter\def\csname PY@tok@sd\endcsname{\let\PY@it=\textit\def\PY@tc##1{\textcolor[rgb]{0.73,0.13,0.13}{##1}}}
\expandafter\def\csname PY@tok@si\endcsname{\let\PY@bf=\textbf\def\PY@tc##1{\textcolor[rgb]{0.73,0.40,0.53}{##1}}}
\expandafter\def\csname PY@tok@se\endcsname{\let\PY@bf=\textbf\def\PY@tc##1{\textcolor[rgb]{0.73,0.40,0.13}{##1}}}
\expandafter\def\csname PY@tok@sr\endcsname{\def\PY@tc##1{\textcolor[rgb]{0.73,0.40,0.53}{##1}}}
\expandafter\def\csname PY@tok@ss\endcsname{\def\PY@tc##1{\textcolor[rgb]{0.10,0.09,0.49}{##1}}}
\expandafter\def\csname PY@tok@sx\endcsname{\def\PY@tc##1{\textcolor[rgb]{0.00,0.50,0.00}{##1}}}
\expandafter\def\csname PY@tok@m\endcsname{\def\PY@tc##1{\textcolor[rgb]{0.40,0.40,0.40}{##1}}}
\expandafter\def\csname PY@tok@gh\endcsname{\let\PY@bf=\textbf\def\PY@tc##1{\textcolor[rgb]{0.00,0.00,0.50}{##1}}}
\expandafter\def\csname PY@tok@gu\endcsname{\let\PY@bf=\textbf\def\PY@tc##1{\textcolor[rgb]{0.50,0.00,0.50}{##1}}}
\expandafter\def\csname PY@tok@gd\endcsname{\def\PY@tc##1{\textcolor[rgb]{0.63,0.00,0.00}{##1}}}
\expandafter\def\csname PY@tok@gi\endcsname{\def\PY@tc##1{\textcolor[rgb]{0.00,0.63,0.00}{##1}}}
\expandafter\def\csname PY@tok@gr\endcsname{\def\PY@tc##1{\textcolor[rgb]{1.00,0.00,0.00}{##1}}}
\expandafter\def\csname PY@tok@ge\endcsname{\let\PY@it=\textit}
\expandafter\def\csname PY@tok@gs\endcsname{\let\PY@bf=\textbf}
\expandafter\def\csname PY@tok@gp\endcsname{\let\PY@bf=\textbf\def\PY@tc##1{\textcolor[rgb]{0.00,0.00,0.50}{##1}}}
\expandafter\def\csname PY@tok@go\endcsname{\def\PY@tc##1{\textcolor[rgb]{0.53,0.53,0.53}{##1}}}
\expandafter\def\csname PY@tok@gt\endcsname{\def\PY@tc##1{\textcolor[rgb]{0.00,0.27,0.87}{##1}}}
\expandafter\def\csname PY@tok@err\endcsname{\def\PY@bc##1{\setlength{\fboxsep}{0pt}\fcolorbox[rgb]{1.00,0.00,0.00}{1,1,1}{\strut ##1}}}
\expandafter\def\csname PY@tok@kc\endcsname{\let\PY@bf=\textbf\def\PY@tc##1{\textcolor[rgb]{0.00,0.50,0.00}{##1}}}
\expandafter\def\csname PY@tok@kd\endcsname{\let\PY@bf=\textbf\def\PY@tc##1{\textcolor[rgb]{0.00,0.50,0.00}{##1}}}
\expandafter\def\csname PY@tok@kn\endcsname{\let\PY@bf=\textbf\def\PY@tc##1{\textcolor[rgb]{0.00,0.50,0.00}{##1}}}
\expandafter\def\csname PY@tok@kr\endcsname{\let\PY@bf=\textbf\def\PY@tc##1{\textcolor[rgb]{0.00,0.50,0.00}{##1}}}
\expandafter\def\csname PY@tok@bp\endcsname{\def\PY@tc##1{\textcolor[rgb]{0.00,0.50,0.00}{##1}}}
\expandafter\def\csname PY@tok@fm\endcsname{\def\PY@tc##1{\textcolor[rgb]{0.00,0.00,1.00}{##1}}}
\expandafter\def\csname PY@tok@vc\endcsname{\def\PY@tc##1{\textcolor[rgb]{0.10,0.09,0.49}{##1}}}
\expandafter\def\csname PY@tok@vg\endcsname{\def\PY@tc##1{\textcolor[rgb]{0.10,0.09,0.49}{##1}}}
\expandafter\def\csname PY@tok@vi\endcsname{\def\PY@tc##1{\textcolor[rgb]{0.10,0.09,0.49}{##1}}}
\expandafter\def\csname PY@tok@vm\endcsname{\def\PY@tc##1{\textcolor[rgb]{0.10,0.09,0.49}{##1}}}
\expandafter\def\csname PY@tok@sa\endcsname{\def\PY@tc##1{\textcolor[rgb]{0.73,0.13,0.13}{##1}}}
\expandafter\def\csname PY@tok@sb\endcsname{\def\PY@tc##1{\textcolor[rgb]{0.73,0.13,0.13}{##1}}}
\expandafter\def\csname PY@tok@sc\endcsname{\def\PY@tc##1{\textcolor[rgb]{0.73,0.13,0.13}{##1}}}
\expandafter\def\csname PY@tok@dl\endcsname{\def\PY@tc##1{\textcolor[rgb]{0.73,0.13,0.13}{##1}}}
\expandafter\def\csname PY@tok@s2\endcsname{\def\PY@tc##1{\textcolor[rgb]{0.73,0.13,0.13}{##1}}}
\expandafter\def\csname PY@tok@sh\endcsname{\def\PY@tc##1{\textcolor[rgb]{0.73,0.13,0.13}{##1}}}
\expandafter\def\csname PY@tok@s1\endcsname{\def\PY@tc##1{\textcolor[rgb]{0.73,0.13,0.13}{##1}}}
\expandafter\def\csname PY@tok@mb\endcsname{\def\PY@tc##1{\textcolor[rgb]{0.40,0.40,0.40}{##1}}}
\expandafter\def\csname PY@tok@mf\endcsname{\def\PY@tc##1{\textcolor[rgb]{0.40,0.40,0.40}{##1}}}
\expandafter\def\csname PY@tok@mh\endcsname{\def\PY@tc##1{\textcolor[rgb]{0.40,0.40,0.40}{##1}}}
\expandafter\def\csname PY@tok@mi\endcsname{\def\PY@tc##1{\textcolor[rgb]{0.40,0.40,0.40}{##1}}}
\expandafter\def\csname PY@tok@il\endcsname{\def\PY@tc##1{\textcolor[rgb]{0.40,0.40,0.40}{##1}}}
\expandafter\def\csname PY@tok@mo\endcsname{\def\PY@tc##1{\textcolor[rgb]{0.40,0.40,0.40}{##1}}}
\expandafter\def\csname PY@tok@ch\endcsname{\let\PY@it=\textit\def\PY@tc##1{\textcolor[rgb]{0.25,0.50,0.50}{##1}}}
\expandafter\def\csname PY@tok@cm\endcsname{\let\PY@it=\textit\def\PY@tc##1{\textcolor[rgb]{0.25,0.50,0.50}{##1}}}
\expandafter\def\csname PY@tok@cpf\endcsname{\let\PY@it=\textit\def\PY@tc##1{\textcolor[rgb]{0.25,0.50,0.50}{##1}}}
\expandafter\def\csname PY@tok@c1\endcsname{\let\PY@it=\textit\def\PY@tc##1{\textcolor[rgb]{0.25,0.50,0.50}{##1}}}
\expandafter\def\csname PY@tok@cs\endcsname{\let\PY@it=\textit\def\PY@tc##1{\textcolor[rgb]{0.25,0.50,0.50}{##1}}}

\def\PYZbs{\char`\\}
\def\PYZus{\char`\_}
\def\PYZob{\char`\{}
\def\PYZcb{\char`\}}
\def\PYZca{\char`\^}
\def\PYZam{\char`\&}
\def\PYZlt{\char`\<}
\def\PYZgt{\char`\>}
\def\PYZsh{\char`\#}
\def\PYZpc{\char`\%}
\def\PYZdl{\char`\$}
\def\PYZhy{\char`\-}
\def\PYZsq{\char`\'}
\def\PYZdq{\char`\"}
\def\PYZti{\char`\~}
% for compatibility with earlier versions
\def\PYZat{@}
\def\PYZlb{[}
\def\PYZrb{]}
\makeatother


    % Exact colors from NB
    \definecolor{incolor}{rgb}{0.0, 0.0, 0.5}
    \definecolor{outcolor}{rgb}{0.545, 0.0, 0.0}



    
    % Prevent overflowing lines due to hard-to-break entities
    \sloppy 
    % Setup hyperref package
    \hypersetup{
      breaklinks=true,  % so long urls are correctly broken across lines
      colorlinks=true,
      urlcolor=urlcolor,
      linkcolor=linkcolor,
      citecolor=citecolor,
      }
    % Slightly bigger margins than the latex defaults
    
    \geometry{verbose,tmargin=1in,bmargin=1in,lmargin=1in,rmargin=1in}
    
    

    \begin{document}
    
    
    \maketitle
    
    

    
    \hypertarget{problema-do-caixeiro-viajante}{%
\section{Problema do
Caixeiro-Viajante}\label{problema-do-caixeiro-viajante}}

O problema do Caixeiro-Viajante trata a seguinte questão:

\emph{Dada uma lista de cidades e a distância entre elas, qual é a rota
mais curta que passa por todas as cidades e retorna à cidade de origem?}

Apesar da curta descrição, este é um dos mais famosos (e complexos)
problemas na Ciência da Computação. Encontrar \emph{a rota mais curta}
entre \(n\) cidades, com algoritmos não otimizados, gasta um tempo
\(O(n!)\): a busca pelo caminho mais curto entre 20 cidades exige muito
mais que o dobro de recursos gastos para encontrar a melhor rota entre
10 cidades.

Usar busca exaustiva entre todos os caminhos garante que o caminho mais
curto será encontrado, mas só é computacionalmente viável para pequenos
conjuntos de cidades. Para problemas maiores, técnicas de otimização são
necessárias para realizar uma busca inteligente no espaço de estados e
encontrar soluções quase-ótimas.

Abaixo, o problema será explicado matemáticamente e uma solução
utilizando busca exaustiva será descrita.

\hypertarget{descriuxe7uxe3o-do-problema-usando-teoria-dos-grafos}{%
\subsubsection{Descrição do problema usando Teoria dos
Grafos}\label{descriuxe7uxe3o-do-problema-usando-teoria-dos-grafos}}

O problema do Caixeiro-Viajante pode ser modelado por grafos não
direcionados com pesos, de modo que as cidades correspondam às vertices,
rotas sejam as arestas, e a distância da rota seja o peso da aresta. A
resposta ótima começa e termina no mesmo vértice, após todos os vértices
terem sidos visitados pelo menos e somente uma vez. Um modelo baseado em
grafos é apresentado na Figura 01.

\includegraphics{imgs/img1.png}

Figura 01 - Exemplo de Mapa usando Grafos. Fonte: PIWONSKA, 2011

\hypertarget{soluuxe7uxe3o-utilizando-algoritmos-de-busca-nuxe3o-informada-i.e.busca-cega}{%
\subsubsection{Solução utilizando Algoritmos de Busca Não-Informada
(i.e.~Busca
Cega)}\label{soluuxe7uxe3o-utilizando-algoritmos-de-busca-nuxe3o-informada-i.e.busca-cega}}

\hypertarget{a-definiuxe7uxe3o-do-escopo-do-problema}{%
\paragraph{a) Definição do Escopo do
Problema}\label{a-definiuxe7uxe3o-do-escopo-do-problema}}

Para que um problema seja resolvido, é necessário definir seguintes
propriedades

\begin{itemize}
\tightlist
\item
  Estados: A descrição do estado corresponde à cidade onde o
  Caixeiro-Viajante se encontra em um dado momento.
\item
  Estado Inicial: Qualquer uma das cidades.
\item
  Ações Possíveis: Ir à uma das cidades que possuem ligação com a cidade
  atual.
\item
  Modelo de Transição: A lista das cidades visitadas.
\item
  Teste de Objetivo: Todas as cidades foram visitadas e a cidade atual é
  a mesma de onde o Caixeiro-Viajante partiu.
\end{itemize}

    \hypertarget{b-estados}{%
\paragraph{b) Estados}\label{b-estados}}

Será utilizado o mapa apresentado na Figura 01. Para representação
computacional, será utilizada uma matriz \(10x10\) onde cada elemento
\(a_{ij}\) indica a distância entre os vértices \(i\) e \(j\). Para
indicar caminhos inexistentes será utilizado o valor \(-1\), e para
indicar a cidade atual será utilizado o valor \(0\).

    \begin{Verbatim}[commandchars=\\\{\}]
{\color{incolor}In [{\color{incolor}1}]:} \PY{c+c1}{\PYZsh{} Deque =\PYZgt{} Double\PYZhy{}Ended Queue }
        \PY{c+c1}{\PYZsh{} Pode ser utilizada tanto como FIFO quanto como LIFO}
        \PY{k+kn}{from} \PY{n+nn}{collections} \PY{k}{import} \PY{n}{deque}
        
        \PY{n}{dist\PYZus{}matrix} \PY{o}{=} \PY{p}{[}
            \PY{p}{[} \PY{l+m+mi}{0}\PY{p}{,}  \PY{l+m+mi}{8}\PY{p}{,} \PY{l+m+mi}{13}\PY{p}{,} \PY{o}{\PYZhy{}}\PY{l+m+mi}{1}\PY{p}{,} \PY{o}{\PYZhy{}}\PY{l+m+mi}{1}\PY{p}{,} \PY{l+m+mi}{14}\PY{p}{,} \PY{o}{\PYZhy{}}\PY{l+m+mi}{1}\PY{p}{,}  \PY{l+m+mi}{8}\PY{p}{,} \PY{o}{\PYZhy{}}\PY{l+m+mi}{1}\PY{p}{,} \PY{o}{\PYZhy{}}\PY{l+m+mi}{1}\PY{p}{]}\PY{p}{,} \PY{c+c1}{\PYZsh{} Cidade 1}
            \PY{p}{[} \PY{l+m+mi}{8}\PY{p}{,}  \PY{l+m+mi}{0}\PY{p}{,}  \PY{l+m+mi}{9}\PY{p}{,} \PY{l+m+mi}{12}\PY{p}{,} \PY{o}{\PYZhy{}}\PY{l+m+mi}{1}\PY{p}{,} \PY{o}{\PYZhy{}}\PY{l+m+mi}{1}\PY{p}{,} \PY{o}{\PYZhy{}}\PY{l+m+mi}{1}\PY{p}{,} \PY{o}{\PYZhy{}}\PY{l+m+mi}{1}\PY{p}{,} \PY{o}{\PYZhy{}}\PY{l+m+mi}{1}\PY{p}{,} \PY{l+m+mi}{11}\PY{p}{]}\PY{p}{,} \PY{c+c1}{\PYZsh{} Cidade 2}
            \PY{p}{[}\PY{l+m+mi}{13}\PY{p}{,}  \PY{l+m+mi}{9}\PY{p}{,}  \PY{l+m+mi}{0}\PY{p}{,} \PY{o}{\PYZhy{}}\PY{l+m+mi}{1}\PY{p}{,} \PY{l+m+mi}{13}\PY{p}{,} \PY{l+m+mi}{15}\PY{p}{,} \PY{o}{\PYZhy{}}\PY{l+m+mi}{1}\PY{p}{,} \PY{o}{\PYZhy{}}\PY{l+m+mi}{1}\PY{p}{,} \PY{o}{\PYZhy{}}\PY{l+m+mi}{1}\PY{p}{,} \PY{o}{\PYZhy{}}\PY{l+m+mi}{1}\PY{p}{]}\PY{p}{,} \PY{c+c1}{\PYZsh{} Cidade 3}
            \PY{p}{[}\PY{o}{\PYZhy{}}\PY{l+m+mi}{1}\PY{p}{,} \PY{l+m+mi}{12}\PY{p}{,} \PY{o}{\PYZhy{}}\PY{l+m+mi}{1}\PY{p}{,}  \PY{l+m+mi}{0}\PY{p}{,} \PY{l+m+mi}{19}\PY{p}{,} \PY{o}{\PYZhy{}}\PY{l+m+mi}{1}\PY{p}{,} \PY{o}{\PYZhy{}}\PY{l+m+mi}{1}\PY{p}{,} \PY{o}{\PYZhy{}}\PY{l+m+mi}{1}\PY{p}{,} \PY{o}{\PYZhy{}}\PY{l+m+mi}{1}\PY{p}{,} \PY{o}{\PYZhy{}}\PY{l+m+mi}{1}\PY{p}{]}\PY{p}{,} \PY{c+c1}{\PYZsh{} Cidade 4}
            \PY{p}{[}\PY{o}{\PYZhy{}}\PY{l+m+mi}{1}\PY{p}{,} \PY{o}{\PYZhy{}}\PY{l+m+mi}{1}\PY{p}{,} \PY{l+m+mi}{13}\PY{p}{,} \PY{l+m+mi}{19}\PY{p}{,}  \PY{l+m+mi}{0}\PY{p}{,} \PY{l+m+mi}{15}\PY{p}{,} \PY{o}{\PYZhy{}}\PY{l+m+mi}{1}\PY{p}{,} \PY{o}{\PYZhy{}}\PY{l+m+mi}{1}\PY{p}{,} \PY{o}{\PYZhy{}}\PY{l+m+mi}{1}\PY{p}{,} \PY{o}{\PYZhy{}}\PY{l+m+mi}{1}\PY{p}{]}\PY{p}{,} \PY{c+c1}{\PYZsh{} Cidade 5}
            \PY{p}{[}\PY{l+m+mi}{14}\PY{p}{,} \PY{o}{\PYZhy{}}\PY{l+m+mi}{1}\PY{p}{,} \PY{l+m+mi}{15}\PY{p}{,} \PY{o}{\PYZhy{}}\PY{l+m+mi}{1}\PY{p}{,} \PY{l+m+mi}{15}\PY{p}{,}  \PY{l+m+mi}{0}\PY{p}{,} \PY{l+m+mi}{22}\PY{p}{,} \PY{l+m+mi}{18}\PY{p}{,} \PY{o}{\PYZhy{}}\PY{l+m+mi}{1}\PY{p}{,} \PY{o}{\PYZhy{}}\PY{l+m+mi}{1}\PY{p}{]}\PY{p}{,} \PY{c+c1}{\PYZsh{} Cidade 6}
            \PY{p}{[}\PY{o}{\PYZhy{}}\PY{l+m+mi}{1}\PY{p}{,} \PY{o}{\PYZhy{}}\PY{l+m+mi}{1}\PY{p}{,} \PY{o}{\PYZhy{}}\PY{l+m+mi}{1}\PY{p}{,} \PY{o}{\PYZhy{}}\PY{l+m+mi}{1}\PY{p}{,} \PY{o}{\PYZhy{}}\PY{l+m+mi}{1}\PY{p}{,} \PY{l+m+mi}{22}\PY{p}{,}  \PY{l+m+mi}{0}\PY{p}{,} \PY{o}{\PYZhy{}}\PY{l+m+mi}{1}\PY{p}{,} \PY{l+m+mi}{21}\PY{p}{,} \PY{o}{\PYZhy{}}\PY{l+m+mi}{1}\PY{p}{]}\PY{p}{,} \PY{c+c1}{\PYZsh{} Cidade 7}
            \PY{p}{[} \PY{l+m+mi}{8}\PY{p}{,} \PY{o}{\PYZhy{}}\PY{l+m+mi}{1}\PY{p}{,} \PY{o}{\PYZhy{}}\PY{l+m+mi}{1}\PY{p}{,} \PY{o}{\PYZhy{}}\PY{l+m+mi}{1}\PY{p}{,} \PY{o}{\PYZhy{}}\PY{l+m+mi}{1}\PY{p}{,} \PY{l+m+mi}{18}\PY{p}{,} \PY{o}{\PYZhy{}}\PY{l+m+mi}{1}\PY{p}{,}  \PY{l+m+mi}{0}\PY{p}{,} \PY{l+m+mi}{10}\PY{p}{,}  \PY{l+m+mi}{8}\PY{p}{]}\PY{p}{,} \PY{c+c1}{\PYZsh{} Cidade 8}
            \PY{p}{[}\PY{o}{\PYZhy{}}\PY{l+m+mi}{1}\PY{p}{,} \PY{o}{\PYZhy{}}\PY{l+m+mi}{1}\PY{p}{,} \PY{o}{\PYZhy{}}\PY{l+m+mi}{1}\PY{p}{,} \PY{o}{\PYZhy{}}\PY{l+m+mi}{1}\PY{p}{,} \PY{o}{\PYZhy{}}\PY{l+m+mi}{1}\PY{p}{,} \PY{o}{\PYZhy{}}\PY{l+m+mi}{1}\PY{p}{,} \PY{l+m+mi}{21}\PY{p}{,} \PY{l+m+mi}{10}\PY{p}{,}  \PY{l+m+mi}{0}\PY{p}{,} \PY{l+m+mi}{12}\PY{p}{]}\PY{p}{,} \PY{c+c1}{\PYZsh{} Cidade 9}
            \PY{p}{[}\PY{o}{\PYZhy{}}\PY{l+m+mi}{1}\PY{p}{,} \PY{l+m+mi}{11}\PY{p}{,} \PY{o}{\PYZhy{}}\PY{l+m+mi}{1}\PY{p}{,} \PY{o}{\PYZhy{}}\PY{l+m+mi}{1}\PY{p}{,} \PY{o}{\PYZhy{}}\PY{l+m+mi}{1}\PY{p}{,} \PY{o}{\PYZhy{}}\PY{l+m+mi}{1}\PY{p}{,} \PY{o}{\PYZhy{}}\PY{l+m+mi}{1}\PY{p}{,}  \PY{l+m+mi}{8}\PY{p}{,} \PY{l+m+mi}{12}\PY{p}{,}  \PY{l+m+mi}{0}\PY{p}{]}\PY{p}{]} \PY{c+c1}{\PYZsh{} Cidade 10}
        
        \PY{n}{list\PYZus{}of\PYZus{}possible\PYZus{}states} \PY{o}{=} \PY{n+nb}{list}\PY{p}{(}\PY{n+nb}{range}\PY{p}{(}\PY{l+m+mi}{1}\PY{p}{,} \PY{n+nb}{len}\PY{p}{(}\PY{n}{dist\PYZus{}matrix}\PY{p}{)} \PY{o}{+} \PY{l+m+mi}{1}\PY{p}{)}\PY{p}{)}
        
        \PY{n+nb}{print}\PY{p}{(}\PY{l+s+s1}{\PYZsq{}}\PY{l+s+s1}{Lista de Estados Possíveis: }\PY{l+s+si}{\PYZpc{}s}\PY{l+s+s1}{ }\PY{l+s+s1}{\PYZsq{}} \PY{o}{\PYZpc{}} \PY{n}{list\PYZus{}of\PYZus{}possible\PYZus{}states}\PY{p}{)}
\end{Verbatim}


    \begin{Verbatim}[commandchars=\\\{\}]
Lista de Estados Possíveis: [1, 2, 3, 4, 5, 6, 7, 8, 9, 10] 

    \end{Verbatim}

    O estado inicial poderá ser qualquer cidade entre 1 e 10

    \begin{Verbatim}[commandchars=\\\{\}]
{\color{incolor}In [{\color{incolor}2}]:} \PY{n}{initial\PYZus{}state} \PY{o}{=} \PY{l+m+mi}{1}
        
        \PY{k}{if} \PY{n}{initial\PYZus{}state} \PY{o}{\PYZlt{}} \PY{l+m+mi}{1} \PY{o+ow}{or} \PY{n}{initial\PYZus{}state} \PY{o}{\PYZgt{}} \PY{n+nb}{len}\PY{p}{(}\PY{n}{dist\PYZus{}matrix}\PY{p}{)}\PY{p}{:}
            \PY{n+nb}{print}\PY{p}{(}\PY{l+s+s1}{\PYZsq{}}\PY{l+s+s1}{ERRO: Estado Inicial Inválido}\PY{l+s+s1}{\PYZsq{}}\PY{p}{)}
        \PY{k}{else}\PY{p}{:}
            \PY{n+nb}{print}\PY{p}{(}\PY{l+s+s1}{\PYZsq{}}\PY{l+s+s1}{Estado Inicial: Cidade }\PY{l+s+si}{\PYZpc{}s}\PY{l+s+s1}{\PYZsq{}} \PY{o}{\PYZpc{}} \PY{n}{initial\PYZus{}state}\PY{p}{)}
\end{Verbatim}


    \begin{Verbatim}[commandchars=\\\{\}]
Estado Inicial: Cidade 1

    \end{Verbatim}

    \hypertarget{c-estruturas-de-dados-e-processamento}{%
\paragraph{c) Estruturas de Dados e
Processamento}\label{c-estruturas-de-dados-e-processamento}}

O primeiro passo para se elaborar uma solução utilizando Busca
Não-Informada é criar um modelo de estrutura de dados para manter o
controle da árvore de busca que será construída.

\hypertarget{c.1-nuxf3s}{%
\subparagraph{c.1) Nós}\label{c.1-nuxf3s}}

O Modelo de Nó descrito no livro \emph{Artificial Intelligente: A Modern
Approach}, faz uso de três atributos: - Estado: O estado no espaço de
estados que gerou este nó - Pai: O nó da árvore que gerou esse nó -
Ação: A ação que foi aplicada para gerar esse nó - Custo do Caminho: O
custo do caminho inicial até este nó

Para descrever um nó, será criada uma classe com todas essas
características.

    \begin{Verbatim}[commandchars=\\\{\}]
{\color{incolor}In [{\color{incolor}3}]:} \PY{k}{class} \PY{n+nc}{Node}\PY{p}{:}
            \PY{l+s+sd}{\PYZdq{}\PYZdq{}\PYZdq{} Classe para representação de Nós \PYZdq{}\PYZdq{}\PYZdq{}}
            
            \PY{k}{def} \PY{n+nf}{\PYZus{}\PYZus{}init\PYZus{}\PYZus{}}\PY{p}{(}\PY{n+nb+bp}{self}\PY{p}{,} \PY{n}{state}\PY{p}{,} \PY{n}{parent}\PY{p}{,} \PY{n}{action}\PY{p}{,} \PY{n}{cost}\PY{p}{)}\PY{p}{:}
                \PY{n+nb+bp}{self}\PY{o}{.}\PY{n}{state} \PY{o}{=} \PY{n}{state}
                \PY{n+nb+bp}{self}\PY{o}{.}\PY{n}{parent} \PY{o}{=} \PY{n}{parent}
                \PY{n+nb+bp}{self}\PY{o}{.}\PY{n}{action} \PY{o}{=} \PY{n}{action}
                \PY{n+nb+bp}{self}\PY{o}{.}\PY{n}{cost} \PY{o}{=} \PY{n}{cost}
        
            \PY{k}{def} \PY{n+nf}{\PYZus{}\PYZus{}repr\PYZus{}\PYZus{}}\PY{p}{(}\PY{n+nb+bp}{self}\PY{p}{)}\PY{p}{:}
                \PY{k}{if} \PY{n+nb+bp}{self}\PY{o}{.}\PY{n}{parent}\PY{p}{:}
                    \PY{k}{return} \PY{l+s+s1}{\PYZsq{}}\PY{l+s+s1}{Node\PYZlt{}State: }\PY{l+s+si}{\PYZpc{}s}\PY{l+s+s1}{, Parent: }\PY{l+s+si}{\PYZpc{}s}\PY{l+s+s1}{, Action: }\PY{l+s+si}{\PYZpc{}s}\PY{l+s+s1}{, Cost: }\PY{l+s+si}{\PYZpc{}s}\PY{l+s+s1}{\PYZgt{}}\PY{l+s+s1}{\PYZsq{}} \PY{o}{\PYZpc{}} \PY{p}{(}\PY{n+nb+bp}{self}\PY{o}{.}\PY{n}{state}\PY{p}{,} \PY{n+nb+bp}{self}\PY{o}{.}\PY{n}{parent}\PY{o}{.}\PY{n}{state}\PY{p}{,} \PY{n+nb+bp}{self}\PY{o}{.}\PY{n}{action}\PY{p}{,} \PY{n+nb+bp}{self}\PY{o}{.}\PY{n}{cost}\PY{p}{)}
                \PY{k}{else}\PY{p}{:}
                    \PY{k}{return} \PY{l+s+s1}{\PYZsq{}}\PY{l+s+s1}{Node\PYZlt{}State: }\PY{l+s+si}{\PYZpc{}s}\PY{l+s+s1}{, Parent: }\PY{l+s+si}{\PYZpc{}s}\PY{l+s+s1}{, Action: }\PY{l+s+si}{\PYZpc{}s}\PY{l+s+s1}{, Cost: }\PY{l+s+si}{\PYZpc{}s}\PY{l+s+s1}{\PYZgt{}}\PY{l+s+se}{\PYZbs{}n}\PY{l+s+s1}{\PYZsq{}} \PY{o}{\PYZpc{}} \PY{p}{(}\PY{n+nb+bp}{self}\PY{o}{.}\PY{n}{state}\PY{p}{,} \PY{l+s+s1}{\PYZsq{}}\PY{l+s+s1}{None}\PY{l+s+s1}{\PYZsq{}}\PY{p}{,} \PY{n+nb+bp}{self}\PY{o}{.}\PY{n}{action}\PY{p}{,} \PY{n+nb+bp}{self}\PY{o}{.}\PY{n}{cost}\PY{p}{)}
\end{Verbatim}


    Dada a estrutura de um Nó, é fácil criar um método que retorne os
componentes para um Nó-Filho. O método \emph{child}\_\emph{node} que
recebe um Nó-Pai, uma ação (que é um inteiro, indicando a cidade de
destino) e o custo da ação atual, para então retornar o Nó-Filho
resultante.

    \begin{Verbatim}[commandchars=\\\{\}]
{\color{incolor}In [{\color{incolor}4}]:} \PY{k}{def} \PY{n+nf}{child\PYZus{}node}\PY{p}{(}\PY{n}{parent}\PY{p}{,} \PY{n}{action}\PY{p}{,} \PY{n}{now\PYZus{}cost}\PY{p}{)}\PY{p}{:}
            \PY{l+s+sd}{\PYZdq{}\PYZdq{}\PYZdq{} Método que retorna um Nó\PYZhy{}Filho à partir de um Nó\PYZhy{}Pai\PYZdq{}\PYZdq{}\PYZdq{}}
            
            \PY{n}{parent\PYZus{}state} \PY{o}{=} \PY{n}{deque}\PY{p}{(}\PY{n}{parent}\PY{o}{.}\PY{n}{state}\PY{p}{)}
            \PY{n}{parent\PYZus{}state}\PY{o}{.}\PY{n}{append}\PY{p}{(}\PY{n}{action}\PY{p}{)}
            
            \PY{n}{child} \PY{o}{=} \PY{n}{Node}\PY{p}{(}\PY{n}{parent\PYZus{}state}\PY{p}{,} \PY{n}{parent}\PY{p}{,} \PY{n}{action}\PY{p}{,} \PY{n}{parent}\PY{o}{.}\PY{n}{cost} \PY{o}{+} \PY{n}{now\PYZus{}cost}\PY{p}{)} 
            
            \PY{k}{return} \PY{n}{child}
\end{Verbatim}


    \hypertarget{c.2-o-problema}{%
\subparagraph{c.2) O problema}\label{c.2-o-problema}}

São necessárias algumas variáveis de controle, além de métodos
específicos ao problema. Para isso, será criada uma classe chamada
\emph{TravellingSalesman} que abstrairá essas estruturas, permitindo que
o Algoritmo de Busca seja genérico, e não específico ao problema: -
Fila: A sequência de ações realizadas, inicialmente vazia. - Estado: O
estado atual. - Teste de Objetivo: Verifica se o objetivo foi cumprido.
- Ações Possíveis: Retorna uma lista com ações possíveis dado o estado
atual.

\textbf{Atenção}: Por não ser uma busca otimizada, não será verificado
se todas as cidades foram visitadas SOMENTE uma vez.

    \begin{Verbatim}[commandchars=\\\{\}]
{\color{incolor}In [{\color{incolor}5}]:} \PY{k}{class} \PY{n+nc}{TravellingSalesman}\PY{p}{:}
            \PY{l+s+sd}{\PYZdq{}\PYZdq{}\PYZdq{} Classe para organização de dados do problema \PYZdq{}\PYZdq{}\PYZdq{}}
            
            \PY{k}{def} \PY{n+nf}{check\PYZus{}goal}\PY{p}{(}\PY{n+nb+bp}{self}\PY{p}{,} \PY{n}{seq}\PY{p}{)}\PY{p}{:}
                \PY{l+s+sd}{\PYZdq{}\PYZdq{}\PYZdq{} Método que retorna verdadeiro caso o objetivo tenha sido alcançado \PYZdq{}\PYZdq{}\PYZdq{}}
                
                \PY{k}{return} \PY{n+nb}{all}\PY{p}{(}\PY{n}{elem} \PY{o+ow}{in} \PY{n}{seq} \PY{k}{for} \PY{n}{elem} \PY{o+ow}{in} \PY{n}{list\PYZus{}of\PYZus{}possible\PYZus{}states}\PY{p}{)} \PY{o+ow}{and} \PY{n}{seq}\PY{p}{[}\PY{n+nb}{len}\PY{p}{(}\PY{n}{seq}\PY{p}{)}\PY{o}{\PYZhy{}}\PY{l+m+mi}{1}\PY{p}{]} \PY{o}{==} \PY{n}{initial\PYZus{}state}
            
            \PY{k}{def} \PY{n+nf}{possible\PYZus{}states}\PY{p}{(}\PY{n+nb+bp}{self}\PY{p}{,} \PY{n}{seq}\PY{p}{)}\PY{p}{:}
                \PY{l+s+sd}{\PYZdq{}\PYZdq{}\PYZdq{} Método que retorna lista de cidades que podem ser acessadas à partir da cidade atual \PYZdq{}\PYZdq{}\PYZdq{}}
                
                \PY{c+c1}{\PYZsh{} Retorna os índices dos estados cujo peso (distância) }
                \PY{c+c1}{\PYZsh{} é maior que 0 (ou seja, o caminho existe e não é ele mesmo)}
                \PY{n}{cur\PYZus{}state} \PY{o}{=} \PY{n}{seq}\PY{p}{[}\PY{n+nb}{len}\PY{p}{(}\PY{n}{seq}\PY{p}{)}\PY{o}{\PYZhy{}}\PY{l+m+mi}{1}\PY{p}{]}
                
                \PY{k}{return} \PY{p}{[}\PY{n}{state}\PY{o}{+}\PY{l+m+mi}{1} \PY{k}{for} \PY{n}{state}\PY{p}{,} \PY{n}{distance} \PY{o+ow}{in} \PY{n+nb}{filter}\PY{p}{(}\PY{k}{lambda} \PY{n}{x}\PY{p}{:} \PY{n}{x}\PY{p}{[}\PY{l+m+mi}{1}\PY{p}{]} \PY{o}{\PYZgt{}} \PY{l+m+mi}{0}\PY{p}{,} \PY{n+nb}{enumerate}\PY{p}{(}\PY{n}{dist\PYZus{}matrix}\PY{p}{[}\PY{n}{cur\PYZus{}state}\PY{o}{\PYZhy{}}\PY{l+m+mi}{1}\PY{p}{]}\PY{p}{)}\PY{p}{)}\PY{p}{]}
\end{Verbatim}


    Forçando elementos na fila para testar se a implementação está correta:

    \begin{Verbatim}[commandchars=\\\{\}]
{\color{incolor}In [{\color{incolor}6}]:} \PY{n}{salesman} \PY{o}{=} \PY{n}{TravellingSalesman}\PY{p}{(}\PY{p}{)}
        
        \PY{n}{seq} \PY{o}{=} \PY{n}{deque}\PY{p}{(}\PY{p}{[}\PY{n}{initial\PYZus{}state}\PY{p}{]}\PY{p}{)}
        
        \PY{n+nb}{print}\PY{p}{(}\PY{l+s+s1}{\PYZsq{}}\PY{l+s+s1}{Lista de Ações: }\PY{l+s+si}{\PYZpc{}s}\PY{l+s+s1}{\PYZsq{}} \PY{o}{\PYZpc{}} \PY{n}{seq}\PY{p}{)}
        \PY{n+nb}{print}\PY{p}{(}\PY{l+s+s1}{\PYZsq{}}\PY{l+s+s1}{Estado Atual: }\PY{l+s+si}{\PYZpc{}s}\PY{l+s+s1}{ }\PY{l+s+s1}{\PYZsq{}} \PY{o}{\PYZpc{}} \PY{n}{seq}\PY{p}{[}\PY{n+nb}{len}\PY{p}{(}\PY{n}{seq}\PY{p}{)}\PY{o}{\PYZhy{}}\PY{l+m+mi}{1}\PY{p}{]}\PY{p}{)}
        \PY{n+nb}{print}\PY{p}{(}\PY{l+s+s1}{\PYZsq{}}\PY{l+s+s1}{Estados Possíveis: }\PY{l+s+si}{\PYZpc{}s}\PY{l+s+s1}{ }\PY{l+s+s1}{\PYZsq{}} \PY{o}{\PYZpc{}} \PY{n}{salesman}\PY{o}{.}\PY{n}{possible\PYZus{}states}\PY{p}{(}\PY{n}{seq}\PY{p}{)}\PY{p}{)}
        \PY{n+nb}{print}\PY{p}{(}\PY{l+s+s1}{\PYZsq{}}\PY{l+s+s1}{Objetivo atingido? }\PY{l+s+si}{\PYZpc{}s}\PY{l+s+s1}{  }\PY{l+s+se}{\PYZbs{}n}\PY{l+s+s1}{\PYZsq{}} \PY{o}{\PYZpc{}} \PY{n}{salesman}\PY{o}{.}\PY{n}{check\PYZus{}goal}\PY{p}{(}\PY{n}{seq}\PY{p}{)}\PY{p}{)}
        
        \PY{n+nb}{print}\PY{p}{(}\PY{l+s+s1}{\PYZsq{}}\PY{l+s+s1}{[...] Inseridos movimentos sequenciais até a cidade 10 }\PY{l+s+se}{\PYZbs{}n}\PY{l+s+s1}{\PYZsq{}}\PY{p}{)}
        \PY{n}{seq} \PY{o}{=} \PY{n}{deque}\PY{p}{(}\PY{p}{[}\PY{l+m+mi}{1}\PY{p}{,}\PY{l+m+mi}{2}\PY{p}{,}\PY{l+m+mi}{3}\PY{p}{,}\PY{l+m+mi}{4}\PY{p}{,}\PY{l+m+mi}{5}\PY{p}{,}\PY{l+m+mi}{6}\PY{p}{,}\PY{l+m+mi}{7}\PY{p}{,}\PY{l+m+mi}{8}\PY{p}{,}\PY{l+m+mi}{9}\PY{p}{,}\PY{l+m+mi}{10}\PY{p}{]}\PY{p}{)}
        
        \PY{n+nb}{print}\PY{p}{(}\PY{l+s+s1}{\PYZsq{}}\PY{l+s+s1}{Lista de Ações: }\PY{l+s+si}{\PYZpc{}s}\PY{l+s+s1}{\PYZsq{}} \PY{o}{\PYZpc{}} \PY{n}{seq}\PY{p}{)}
        \PY{n+nb}{print}\PY{p}{(}\PY{l+s+s1}{\PYZsq{}}\PY{l+s+s1}{Estado Atual: }\PY{l+s+si}{\PYZpc{}s}\PY{l+s+s1}{ }\PY{l+s+s1}{\PYZsq{}} \PY{o}{\PYZpc{}} \PY{n}{seq}\PY{p}{[}\PY{n+nb}{len}\PY{p}{(}\PY{n}{seq}\PY{p}{)}\PY{o}{\PYZhy{}}\PY{l+m+mi}{1}\PY{p}{]}\PY{p}{)}
        \PY{n+nb}{print}\PY{p}{(}\PY{l+s+s1}{\PYZsq{}}\PY{l+s+s1}{Estados Possíveis: }\PY{l+s+si}{\PYZpc{}s}\PY{l+s+s1}{ }\PY{l+s+s1}{\PYZsq{}} \PY{o}{\PYZpc{}} \PY{n}{salesman}\PY{o}{.}\PY{n}{possible\PYZus{}states}\PY{p}{(}\PY{n}{seq}\PY{p}{)}\PY{p}{)}
        \PY{n+nb}{print}\PY{p}{(}\PY{l+s+s1}{\PYZsq{}}\PY{l+s+s1}{Objetivo atingido? }\PY{l+s+si}{\PYZpc{}s}\PY{l+s+s1}{  }\PY{l+s+se}{\PYZbs{}n}\PY{l+s+s1}{\PYZsq{}} \PY{o}{\PYZpc{}} \PY{n}{salesman}\PY{o}{.}\PY{n}{check\PYZus{}goal}\PY{p}{(}\PY{n}{seq}\PY{p}{)}\PY{p}{)}
        
        \PY{n+nb}{print}\PY{p}{(}\PY{l+s+s1}{\PYZsq{}}\PY{l+s+s1}{[...] Inseridos movimentos de retorno até a cidade 1 }\PY{l+s+se}{\PYZbs{}n}\PY{l+s+s1}{\PYZsq{}}\PY{p}{)}
        \PY{n}{seq} \PY{o}{=} \PY{n}{deque}\PY{p}{(}\PY{p}{[}\PY{l+m+mi}{2}\PY{p}{,}\PY{l+m+mi}{3}\PY{p}{,}\PY{l+m+mi}{4}\PY{p}{,}\PY{l+m+mi}{5}\PY{p}{,}\PY{l+m+mi}{6}\PY{p}{,}\PY{l+m+mi}{7}\PY{p}{,}\PY{l+m+mi}{8}\PY{p}{,}\PY{l+m+mi}{9}\PY{p}{,}\PY{l+m+mi}{10}\PY{p}{,}\PY{l+m+mi}{9}\PY{p}{,}\PY{l+m+mi}{8}\PY{p}{,}\PY{l+m+mi}{7}\PY{p}{,}\PY{l+m+mi}{6}\PY{p}{,}\PY{l+m+mi}{5}\PY{p}{,}\PY{l+m+mi}{4}\PY{p}{,}\PY{l+m+mi}{3}\PY{p}{,}\PY{l+m+mi}{2}\PY{p}{,}\PY{l+m+mi}{1}\PY{p}{]}\PY{p}{)}
        
        \PY{n+nb}{print}\PY{p}{(}\PY{l+s+s1}{\PYZsq{}}\PY{l+s+s1}{Lista de Ações: }\PY{l+s+si}{\PYZpc{}s}\PY{l+s+s1}{\PYZsq{}} \PY{o}{\PYZpc{}} \PY{n}{seq}\PY{p}{)}
        \PY{n+nb}{print}\PY{p}{(}\PY{l+s+s1}{\PYZsq{}}\PY{l+s+s1}{Estado Atual: }\PY{l+s+si}{\PYZpc{}s}\PY{l+s+s1}{ }\PY{l+s+s1}{\PYZsq{}} \PY{o}{\PYZpc{}} \PY{n}{seq}\PY{p}{[}\PY{n+nb}{len}\PY{p}{(}\PY{n}{seq}\PY{p}{)}\PY{o}{\PYZhy{}}\PY{l+m+mi}{1}\PY{p}{]}\PY{p}{)}
        \PY{n+nb}{print}\PY{p}{(}\PY{l+s+s1}{\PYZsq{}}\PY{l+s+s1}{Estados Possíveis: }\PY{l+s+si}{\PYZpc{}s}\PY{l+s+s1}{ }\PY{l+s+s1}{\PYZsq{}} \PY{o}{\PYZpc{}} \PY{n}{salesman}\PY{o}{.}\PY{n}{possible\PYZus{}states}\PY{p}{(}\PY{n}{seq}\PY{p}{)}\PY{p}{)}
        \PY{n+nb}{print}\PY{p}{(}\PY{l+s+s1}{\PYZsq{}}\PY{l+s+s1}{Objetivo atingido? }\PY{l+s+si}{\PYZpc{}s}\PY{l+s+s1}{  }\PY{l+s+se}{\PYZbs{}n}\PY{l+s+s1}{\PYZsq{}} \PY{o}{\PYZpc{}} \PY{n}{salesman}\PY{o}{.}\PY{n}{check\PYZus{}goal}\PY{p}{(}\PY{n}{seq}\PY{p}{)}\PY{p}{)}
        
        \PY{c+c1}{\PYZsh{} Limpar variáveis para não influenciar o próximo algoritmo}
        \PY{n}{salesman} \PY{o}{=} \PY{k+kc}{None}
        \PY{n}{seq} \PY{o}{=} \PY{k+kc}{None}
\end{Verbatim}


    \begin{Verbatim}[commandchars=\\\{\}]
Lista de Ações: deque([1])
Estado Atual: 1 
Estados Possíveis: [2, 3, 6, 8] 
Objetivo atingido? False  

[{\ldots}] Inseridos movimentos sequenciais até a cidade 10 

Lista de Ações: deque([1, 2, 3, 4, 5, 6, 7, 8, 9, 10])
Estado Atual: 10 
Estados Possíveis: [2, 8, 9] 
Objetivo atingido? False  

[{\ldots}] Inseridos movimentos de retorno até a cidade 1 

Lista de Ações: deque([2, 3, 4, 5, 6, 7, 8, 9, 10, 9, 8, 7, 6, 5, 4, 3, 2, 1])
Estado Atual: 1 
Estados Possíveis: [2, 3, 6, 8] 
Objetivo atingido? True  


    \end{Verbatim}

    \hypertarget{busca-nuxe3o-informada}{%
\subsubsection{Busca Não-Informada}\label{busca-nuxe3o-informada}}

O termo Busca Não-Informada (ou Busca Cega) indica que não são
utilizadas informações além daquelas apresentadas pelo problema. Tudo
que o agente pode fazer é verificar se o objetivo foi ou não atingido.

Existem dois algoritmos principais de Busca Não-Informada: Busca em
Largura e Busca em Profundidade. Na Busca em Largura, todos os nós
possíveis são expandidos igualmente, e verifica-se se o resultado foi
atingido. Na Busca em Profundidade, cada nó é expandido individualmente
até que alguma condição de falha seja atingida, e só então o próximo nó
aberto (fronteira) é avaliado.

\hypertarget{criauxe7uxe3o-do-nuxf3-inicial}{%
\paragraph{Criação do Nó Inicial}\label{criauxe7uxe3o-do-nuxf3-inicial}}

Independente do algoritmo de busca não-informada utilizado, este começa
à partir de um Nó Inicial que contém as informações do estado inicial.

\textbf{Atenção}: Neste problema, a resolução depende de todos os
estados passados, portanto o estado armazenado no Nó corresponde ao
conjunto dos estados passados, para simplificar o algoritmo de
verificação. Outra possibilidade, apresentada no livro \emph{Artificial
Intelligence: A Modern Approach} para avaliar se o objetivo foi
atingido, seria analisar recursivamente os Nós da árvore, o que
diminuiria o uso de memória para armazenamento da árvore mas aumentaria
o tempo de processamento.

    \hypertarget{busca-em-largura-breadth-first-search-bfs}{%
\paragraph{\texorpdfstring{Busca em Largura (\emph{Breadth-First Search
{[}BFS{]}})}{Busca em Largura (Breadth-First Search {[}BFS{]})}}\label{busca-em-largura-breadth-first-search-bfs}}

Esta é uma estratégia simples onde o nó inicial é expandido primeiro e
todos os seus sucessores diretos são expandidos em seguida.

    \begin{Verbatim}[commandchars=\\\{\}]
{\color{incolor}In [{\color{incolor}7}]:} \PY{n}{salesman} \PY{o}{=} \PY{n}{TravellingSalesman}\PY{p}{(}\PY{p}{)}
        
        \PY{n}{seq} \PY{o}{=} \PY{n}{deque}\PY{p}{(}\PY{p}{[}\PY{n}{initial\PYZus{}state}\PY{p}{]}\PY{p}{)}
        \PY{n}{node} \PY{o}{=} \PY{n}{Node}\PY{p}{(}\PY{n}{seq}\PY{p}{,} \PY{k+kc}{None}\PY{p}{,} \PY{k+kc}{None}\PY{p}{,} \PY{l+m+mi}{0}\PY{p}{)}
        
        \PY{n}{frontier} \PY{o}{=} \PY{n}{deque}\PY{p}{(}\PY{p}{[}\PY{n}{node}\PY{p}{]}\PY{p}{)}
        \PY{n}{explored} \PY{o}{=} \PY{n}{deque}\PY{p}{(}\PY{p}{)}
        
        \PY{k}{while} \PY{k+kc}{True}\PY{p}{:}    
            \PY{k}{if} \PY{n}{salesman}\PY{o}{.}\PY{n}{check\PYZus{}goal}\PY{p}{(}\PY{n}{node}\PY{o}{.}\PY{n}{state}\PY{p}{)}\PY{p}{:}
                \PY{n+nb}{print}\PY{p}{(}\PY{l+s+s1}{\PYZsq{}}\PY{l+s+s1}{Solução Encontrada: }\PY{l+s+si}{\PYZpc{}s}\PY{l+s+s1}{\PYZsq{}} \PY{o}{\PYZpc{}} \PY{n}{node}\PY{o}{.}\PY{n}{state}\PY{p}{)}
                \PY{k}{break}
        
            \PY{k}{elif} \PY{n+nb}{len}\PY{p}{(}\PY{n}{frontier}\PY{p}{)} \PY{o}{==} \PY{l+m+mi}{0}\PY{p}{:} 
                \PY{n+nb}{print}\PY{p}{(}\PY{l+s+s1}{\PYZsq{}}\PY{l+s+s1}{Todos os Nós Explorados. Nenhuma Rota Encontrada}\PY{l+s+s1}{\PYZsq{}}\PY{p}{)}
                \PY{k}{break}
        
            \PY{k}{else}\PY{p}{:}
                \PY{c+c1}{\PYZsh{} PopLeft \PYZhy{}\PYZgt{} Remove o Nó mais à esquerda (FIFO)}
                \PY{n}{node} \PY{o}{=} \PY{n}{frontier}\PY{o}{.}\PY{n}{popleft}\PY{p}{(}\PY{p}{)}
                
                \PY{c+c1}{\PYZsh{} State é a lista de estados visitado. O Estado Atual é o último estado.}
                \PY{n}{cur\PYZus{}state} \PY{o}{=} \PY{n}{node}\PY{o}{.}\PY{n}{state}\PY{p}{[}\PY{n+nb}{len}\PY{p}{(}\PY{n}{node}\PY{o}{.}\PY{n}{state}\PY{p}{)}\PY{o}{\PYZhy{}}\PY{l+m+mi}{1}\PY{p}{]}
            
                \PY{c+c1}{\PYZsh{} Append \PYZhy{}\PYZgt{} Adiciona o elemento à direita.}
                \PY{n}{explored}\PY{o}{.}\PY{n}{append}\PY{p}{(}\PY{n}{cur\PYZus{}state}\PY{p}{)}
        
                \PY{k}{for} \PY{n}{state} \PY{o+ow}{in} \PY{n}{salesman}\PY{o}{.}\PY{n}{possible\PYZus{}states}\PY{p}{(}\PY{n}{node}\PY{o}{.}\PY{n}{state}\PY{p}{)}\PY{p}{:}
                    \PY{n}{new\PYZus{}node} \PY{o}{=} \PY{n}{child\PYZus{}node}\PY{p}{(}\PY{n}{node}\PY{p}{,} \PY{n}{state}\PY{p}{,} \PY{n}{dist\PYZus{}matrix}\PY{p}{[}\PY{n}{cur\PYZus{}state}\PY{o}{\PYZhy{}}\PY{l+m+mi}{1}\PY{p}{]}\PY{p}{[}\PY{n}{state}\PY{o}{\PYZhy{}}\PY{l+m+mi}{1}\PY{p}{]}\PY{p}{)}
                    \PY{n}{frontier}\PY{o}{.}\PY{n}{append}\PY{p}{(}\PY{n}{new\PYZus{}node}\PY{p}{)}
        \PY{c+c1}{\PYZsh{} Limpar variáveis para não influenciar o próximo algoritmo}
        \PY{n}{salesman} \PY{o}{=} \PY{k+kc}{None}
        \PY{n}{seq} \PY{o}{=} \PY{k+kc}{None}
        \PY{n}{node} \PY{o}{=} \PY{k+kc}{None}
        \PY{n}{frontier} \PY{o}{=} \PY{k+kc}{None}
        \PY{n}{explored} \PY{o}{=} \PY{k+kc}{None}
\end{Verbatim}


    \begin{Verbatim}[commandchars=\\\{\}]
Solução Encontrada: deque([1, 2, 4, 5, 3, 6, 7, 9, 10, 8, 1])

    \end{Verbatim}

    \hypertarget{busca-em-profundidade-depht-first-search-dfs}{%
\paragraph{Busca em Profundidade (Depht-First Search
{[}DFS{]})}\label{busca-em-profundidade-depht-first-search-dfs}}

    Nesta estratégia de busca, o nó é expandido até que chegue ao fim (ou
atinga uma profundidade pré-determinada).

    \begin{Verbatim}[commandchars=\\\{\}]
{\color{incolor}In [{\color{incolor}8}]:} \PY{n}{salesman} \PY{o}{=} \PY{n}{TravellingSalesman}\PY{p}{(}\PY{p}{)}
        
        \PY{n}{seq} \PY{o}{=} \PY{n}{deque}\PY{p}{(}\PY{p}{[}\PY{n}{initial\PYZus{}state}\PY{p}{]}\PY{p}{)}
        \PY{n}{node} \PY{o}{=} \PY{n}{Node}\PY{p}{(}\PY{n}{seq}\PY{p}{,} \PY{k+kc}{None}\PY{p}{,} \PY{k+kc}{None}\PY{p}{,} \PY{l+m+mi}{0}\PY{p}{)}
        \PY{n}{num\PYZus{}iter} \PY{o}{=} \PY{l+m+mi}{0}
        
        \PY{k}{def} \PY{n+nf}{recursive\PYZus{}dfs}\PY{p}{(}\PY{n}{node}\PY{p}{,} \PY{n}{limit}\PY{p}{)}\PY{p}{:}
            \PY{k}{if} \PY{n}{salesman}\PY{o}{.}\PY{n}{check\PYZus{}goal}\PY{p}{(}\PY{n}{node}\PY{o}{.}\PY{n}{state}\PY{p}{)}\PY{p}{:}
                \PY{k}{return} \PY{p}{(}\PY{l+s+s1}{\PYZsq{}}\PY{l+s+s1}{Solução Encontrada: }\PY{l+s+si}{\PYZpc{}s}\PY{l+s+s1}{\PYZsq{}} \PY{o}{\PYZpc{}} \PY{n}{node}\PY{o}{.}\PY{n}{state}\PY{p}{)}
            
            \PY{k}{elif} \PY{n}{limit} \PY{o}{==} \PY{l+m+mi}{0}\PY{p}{:}
                \PY{k}{return} \PY{l+s+s2}{\PYZdq{}}\PY{l+s+s2}{cutoff}\PY{l+s+s2}{\PYZdq{}}
        
            \PY{k}{else}\PY{p}{:}       
                \PY{n}{cutoff\PYZus{}ocurred} \PY{o}{=} \PY{k+kc}{False}
                
                \PY{k}{for} \PY{n}{state} \PY{o+ow}{in} \PY{n}{salesman}\PY{o}{.}\PY{n}{possible\PYZus{}states}\PY{p}{(}\PY{n}{node}\PY{o}{.}\PY{n}{state}\PY{p}{)}\PY{p}{:}
                    \PY{n}{new\PYZus{}node} \PY{o}{=} \PY{n}{child\PYZus{}node}\PY{p}{(}\PY{n}{node}\PY{p}{,} \PY{n}{state}\PY{p}{,} \PY{n}{dist\PYZus{}matrix}\PY{p}{[}\PY{n}{cur\PYZus{}state}\PY{o}{\PYZhy{}}\PY{l+m+mi}{1}\PY{p}{]}\PY{p}{[}\PY{n}{state}\PY{o}{\PYZhy{}}\PY{l+m+mi}{1}\PY{p}{]}\PY{p}{)}
                    \PY{n}{result} \PY{o}{=} \PY{n}{recursive\PYZus{}dfs}\PY{p}{(}\PY{n}{new\PYZus{}node}\PY{p}{,} \PY{n}{limit}\PY{o}{\PYZhy{}}\PY{l+m+mi}{1}\PY{p}{)}
                    
                    \PY{k}{if} \PY{n}{result} \PY{o}{==} \PY{l+s+s2}{\PYZdq{}}\PY{l+s+s2}{cutoff}\PY{l+s+s2}{\PYZdq{}}\PY{p}{:}
                        \PY{n}{cutoff\PYZus{}ocurred} \PY{o}{=} \PY{k+kc}{True}
                    \PY{k}{else}\PY{p}{:}
                        \PY{k}{return} \PY{n}{result}
                
                \PY{k}{if} \PY{n}{cutoff\PYZus{}ocurred}\PY{p}{:}
                    \PY{k}{return} \PY{l+s+s2}{\PYZdq{}}\PY{l+s+s2}{cutoff}\PY{l+s+s2}{\PYZdq{}}
            
        \PY{n+nb}{print}\PY{p}{(}\PY{n}{recursive\PYZus{}dfs}\PY{p}{(}\PY{n}{node}\PY{p}{,} \PY{l+m+mi}{10}\PY{p}{)}\PY{p}{)}
\end{Verbatim}


    \begin{Verbatim}[commandchars=\\\{\}]
Solução Encontrada: deque([1, 2, 4, 5, 3, 6, 7, 9, 10, 8, 1])

    \end{Verbatim}


    % Add a bibliography block to the postdoc
    
    
    
    \end{document}
