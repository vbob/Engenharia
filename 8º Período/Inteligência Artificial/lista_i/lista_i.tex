\documentclass[tikz, paper=a4, fontsize=11pt]{scrartcl} % A4 paper and 11pt font size

    \usepackage[T1]{fontenc} % Use 8-bit encoding that has 256 glyphs
    \usepackage{fourier} % Use the Adobe Utopia font for the document - comment this line to return to the LaTeX default
    \usepackage[english]{babel} % English language/hyphenation
    \usepackage{amsmath,amsfonts,amsthm} % Math packages

    \usepackage{tikz}
    \usepackage{verbatim}
    \usetikzlibrary{graphdrawing}
    \usetikzlibrary{graphs}
    \usegdlibrary{trees}
    
    \usepackage{lipsum} % Used for inserting dummy 'Lorem ipsum' text into the template
    
    \usepackage{sectsty} % Allows customizing section commands
    \allsectionsfont{\normalfont\scshape} % Make all sections centered, the default font and small caps
    
    \usepackage{fancyhdr} % Custom headers and footers
    \pagestyle{fancyplain} % Makes all pages in the document conform to the custom headers and footers
    \fancyhead{} % No page header - if you want one, create it in the same way as the footers below
    \fancyfoot[L]{} % Empty left footer
    \fancyfoot[C]{} % Empty center footer
    \fancyfoot[R]{\thepage} % Page numbering for right footer
    \renewcommand{\headrulewidth}{0pt} % Remove header underlines
    \renewcommand{\footrulewidth}{0pt} % Remove footer underlines
    \setlength{\headheight}{13.6pt} % Customize the height of the header
    
    \numberwithin{equation}{section} % Number equations within sections (i.e. 1.1, 1.2, 2.1, 2.2 instead of 1, 2, 3, 4)
    \numberwithin{figure}{section} % Number figures within sections (i.e. 1.1, 1.2, 2.1, 2.2 instead of 1, 2, 3, 4)
    \numberwithin{table}{section} % Number tables within sections (i.e. 1.1, 1.2, 2.1, 2.2 instead of 1, 2, 3, 4)
    
    \setlength\parindent{0pt} % Removes all indentation from paragraphs - comment this line for an assignment with lots of text

    %----------------------------------------------------------------------------------------
    %	TITLE SECTION
    %----------------------------------------------------------------------------------------
    
    \newcommand{\horrule}[1]{\rule{\linewidth}{#1}} % Create horizontal rule command with 1 argument of height
    
    \title{	
    \normalfont \normalsize 
    \textsc{Instituto Federal de Educação, Ciência e Tecnologia de Mato Grosso} \\ [25pt] % Your university, school and/or department name(s)
    \horrule{0.5pt} \\[0.4cm] % Thin top horizontal rule
    \huge Inteligência Artificial \\ % The assignment title
    \huge Lista I - Solução por Meio de Buscas
    \horrule{2pt} \\[0.5cm] % Thick bottom horizontal rule
    }
    
    \author{Vitor Bruno de Oliveira Barth} % Your name
    
    \date{\normalsize\today} % Today's date or a custom date
    
    \begin{document}
    
    \maketitle
    
    %----------------------------------------------------------------------------------------
    %	Exercício 1
    %----------------------------------------------------------------------------------------
    
    \section{Nomeie os algoritmos de buscas que resultam de:}
    \begin{enumerate}
        \item Busca de Feixe Local, com k = 1 
        \par R: 

        \item Busca de Feixe Local, com um estado inicial e sem limite de estados retidos
        \par R:

        \item Busca de Têmpera Simulada, com temperatura T = $\infty$ todo o tempo
        \par R:
    \end{enumerate}

     %----------------------------------------------------------------------------------------
    %	Exercício 1
    %----------------------------------------------------------------------------------------
    
    \section{Considerando a árvore abaixo, pede-se a ordem com que os nós são visitados usando-se os seguintes algoritmos de busca:}
    \begin{enumerate}
        \item Busca de Feixe Local, com k = 1 
        \par R: 

        \item Busca de Feixe Local, com um estado inicial e sem limite de estados retidos
        \par R:

        \item Busca de Têmpera Simulada, com temperatura T = $\infty$ todo o tempo
        \par R:
    \end{enumerate}
    
    \end{document}