\documentclass[paper=a4, fontsize=11pt]{scrartcl} % A4 paper and 11pt font size
    \usepackage{geometry}
    \geometry{
    a4paper,
    total={170mm,257mm},
    left=20mm,
    top=20mm,
    }

    \usepackage[T1]{fontenc} % Use 8-bit encoding that has 256 glyphs
    \usepackage{fourier} % Use the Adobe Utopia font for the document - comment this line to return to the LaTeX default
    \usepackage[brazil]{babel} % English language/hyphenation
    \usepackage{amsmath,amsfonts,amsthm} % Math packages
    \usepackage{enumerate}
    \usepackage{tikz}
    \usetikzlibrary{graphs}

    \usepackage{sectsty} % Allows customizing section commands
    \allsectionsfont{\normalfont\scshape} % Make all sections centered, the default font and small caps

    \setlength\parindent{0pt} % Removes all indentation from paragraphs - comment this line for an assignment with lots of text

    %----------------------------------------------------------------------------------------
    %	TITLE SECTION
    %----------------------------------------------------------------------------------------
    
    \newcommand{\horrule}[1]{\rule{\linewidth}{#1}} % Create horizontal rule command with 1 argument of height
    
    \title{	
    \normalfont \normalsize 
    \textsc{Instituto Federal de Educação, Ciência e Tecnologia de Mato Grosso} \\ [25pt] % Your university, school and/or department name(s)
    \horrule{0.5pt} \\[0.4cm] % Thin top horizontal rule
    \huge Inteligência Artificial \\ % The assignment title
    \huge Lista I - Solução por Meio de Buscas
    \horrule{2pt} \\[0.5cm] % Thick bottom horizontal rule
    }
    
    \author{Vitor Bruno de Oliveira Barth} % Your name
    
    \date{\normalsize\today} % Today's date or a custom date
    
    \begin{document}
    
    \maketitle
    
    %----------------------------------------------------------------------------------------
    %	Exercício 1
    %----------------------------------------------------------------------------------------
    
    \section{Nomeie os algoritmos de buscas que resultam de:}
    \begin{enumerate}[(a)]
        \item Busca de Feixe Local, com k = 1 
        \par R: 

        \item Busca de Feixe Local, com um estado inicial e sem limite de estados retidos
        \par R:

        \item Busca de Têmpera Simulada, com temperatura T = $\infty$ todo o tempo
        \par R:
    \end{enumerate}

    %----------------------------------------------------------------------------------------
    %	Exercício 2
    %----------------------------------------------------------------------------------------
    
    \section{Considerando a árvore abaixo, pede-se a ordem com que os nós são visitados usando-se os seguintes algoritmos de busca:}
    \hfil
\begin{tikzpicture}[
    every node/.style = {shape=circle, draw, align=center}
    ]
    \node {1}
    child { node {2} 
        child { node {5} 
            child { node {11} }
            child { node {12} }
        }
        child [missing]
        child { node {6} 
            child { node {13} }
        }
    }
    child [missing]
    child [missing]
    child {node {3} 
        child { node {7} }
        child { node {8} }
    }
    child [missing]
    child [missing]
    child {node {4} 
        child { node {9} 
            child { node {14} }
            child { node {15} }
        }
        child [missing]
        child { node {10} 
            child { node {16} }
            child { node {17} }
        }
    };
  \end{tikzpicture}

    \begin{enumerate}[(a)]
        \item Breadth-First Search
        \par R: 

        \item Depth-First Search
        \par R:

        \item Depth-First Iterative-Deepening Search
        \par R:
    \end{enumerate}

    %----------------------------------------------------------------------------------------
    %	Exercício 3
    %----------------------------------------------------------------------------------------
    
    \section{Desenhe a árvore de busca completa (do Estado $S$ ao $G$) para o grafo abaixo. Os números ao lado dos nós representam as distâncias estimadas do Estado Inicial ($S$) para o Estado Final ($G$). Mostre como o procedimento de buscas procede na árvore quando usando:}
    \hfil
\begin{tikzpicture}
    \begin{scope}[grow'=right,
        every node/.style = {shape=circle, draw, align=center},
        every tree node/.style={anchor=base west}]
        \node[label={\small 11}] (S) at (0,0) {S};
        \node[label={\small 8}] (A) at (1,1) {A};
        \node[label=below:{\small 9}] (B) at (1,-1) {B};
        \node[label=below:{\small 10}] (C) at (3,-1) {C};
        \node[label={\small 5}] (D) at (3,1) {D};
        \node[label={\small 0}] (G) at (4,0) {G};
    \end{scope}

    \begin{scope}
    \path [-] (S) edge node {} (A);
    \path [-] (S) edge node {} (B);
    \path [-] (A) edge node {} (B);
    \path [-] (B) edge node {} (C);
    \path [-] (A) edge node {} (D);
    \path [-] (B) edge node {} (D);
    \path [-] (D) edge node {} (G);
    \end{scope}

  \end{tikzpicture}

    \begin{enumerate}[(a)]
        \item DFS - Busca em Profundidade
        \par R: 

        \item BFS - Busca em Largura
        \par R:

        \item HC - Subida de Encosta
        \par R:

        \item BS - Feixe-Local com $k=2$
        \par R:

        \item GBFS - Busca Gulosa
        \par R:

    \end{enumerate}


    \end{document}