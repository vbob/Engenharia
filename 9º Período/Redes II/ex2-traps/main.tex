%%%%%%%%%%%%%%%%%%%%%%%%%%%%%%%%%%%%%%%%%
%----------------------------------------------------------------------------------------
%	PACKAGES AND OTHER DOCUMENT CONFIGURATIONS
%----------------------------------------------------------------------------------------

\documentclass{article}

\input{structure.tex} % Include the file specifying the document structure and custom commands

%----------------------------------------------------------------------------------------
%	ASSIGNMENT INFORMATION
%----------------------------------------------------------------------------------------

\title{REDES II: Configuração de Traps SNMP} % Title of the assignment

\author{Vitor Bruno de Oliveira Barth\\ \texttt{vbob@vbob.com.br}} % Author name and email address

\date{Instituto Federal de Educação, Ciência e Tecnologia de Mato Grosso --- \today} % University, school and/or department name(s) and a date

%----------------------------------------------------------------------------------------

\begin{document}

\maketitle % Print the title

%----------------------------------------------------------------------------------------
%	INTRODUCTION
%----------------------------------------------------------------------------------------

\section{Introdução} % Unnumbered section

Traps SNMP são amplamente utilizadas para o Gerenciamento de Redes. Elas permitem que o Agente comunique ao Gerente atividades indevidas, falhas ou quaisquer outras ocorrências que devam ser analisdas.

Neste trabalho, é apresentada a forma de configuração de duas traps:

\begin{enumerate}
\item Mudança de estado de uma interface de rede para \textit{down}.
\item Acesso indevido via SSH
\end{enumerate}

Ao ser recebida pelo gerente, cada uma dessas traps executará um \textit{script}, que servirá para avisar ao usuário do comportamento detectado.

%----------------------------------------------------------------------------------------
%	PROBLEM 1
%----------------------------------------------------------------------------------------

\section{Configuração das Traps} % Numbered section

\par Para a configuração das Traps, são utilizadas duas MIBs padrão do pacote \textit{net-snmp}:

\begin{enumerate}
\item \textbf{IF-MIB::linkDown} para indicar que falha em uma conexão de rede
\item \textbf{SNMPv2-MIB::authenticationFailure} para indicar falha na conexão via SSH
\end{enumerate}

Estas MIBs foram projetdas justamente para este propósito, e por esta razão é desnecessária a criação de MIBs adicionais.

\subsection{Configuração dos Gatilhos}

\par Ao ser recebida, cada trap deverá executar um \textit{script shell} que realizará ações para avisar ao usuário sobre a chamada da \textit{trap}.

\par A definição de tais gatilhos é realizada no arquivo \textit{/etc/snmp/snmptrapd.conf}. O arquivo editado é apresentado abaixo:

\begin{program}
\begin{myverb}
disableAuthorization yes
authCommunity log,execute,net public
outputOption n

traphandle IF-MIB::linkDown /usr/local/etc/snmptrap/linkDown.sh
traphandle SNMPv2-MIB::authenticationFailure /usr/local/etc/snmptrap/authenticationFailure.sh 
\end{myverb}
\caption{/etc/snmp/snmptrapd.conf}
\end{program}

\par O comando \textit{disableAuthorization} indica que não é necessária autorização para que uma \textit{trap} seja processada.
\par O comando \textit{authCommunity log,execute,net public} permite à comunidade \textit{public} seja utilizada para o envio de \textit{traps}.
\par O comando \textit{outputOption n} indica que as OIDs serão armazenadas de forma numérica.

\pagebreak

\par Após ser editado, o \textit{daemon} deve ser reiniciado, através do comando

\begin{commandline}
\begin{verbatim}
# systemctl restart snmptrapd
\end{verbatim}
\end{commandline}

\par Deste modo, o servidor estará ouvindo às \textit{traps} e configurado para executar os \textit{scripts} desejados.

%----------------------------------------------------------------------------------------------------
%	PROBLEM 2
%----------------------------------------------------------------------------------------

\section{\textit{Scripts}}

Com os \textit{scripts} definidos no arquivo \textit{/etc/snmp/snmptrapd.conf}, estes devem ser construídos para que sejam executadas ações para avisar o usuário das ocorrências.

\par No caso da \textit{trap} \textbf{IF-MIB::linkDown} será enviado um e-mail ao administrador da rede.
\par Ao ser recebida a \textit{trap} \textbf{SMPv2-MIB::authenticationFailure}, esta ocorrência será salva em um arquivo, onde estarão as últimas 10 tentativas mal sucedidas de \textit{login}.

\subsection{IF-MIB::linkDown}

\par Para se enviar um e-mail utilizando \textit{Linux} pode ser utilizado o pacote \textit{sendmail}. O \textit{script} completo está apresentado abaixo.

\begin{program}
\begin{lstlisting}[language=bash]
#!/bin/bash

email=sysadmin@enterprise.com
file=/usr/local/etc/snmptrap/email.txt

sendmail $email < $file
\end{lstlisting}
\caption{/usr/local/etc/snmptrap/linkDown.sh}
\end{program}

\par Onde \textit{email.txt} é o corpo da mensagem que se deseja ter enviada.

\subsection{SNMPv2-MIB::authenticationFailure}

Para se armazenar em arquivo as últimas 10 tentativas de \textit{login} realizadas por SSH, deve-se obter algumas informações contidas na mensagem enviada pela \textit{trap}. Para isso, será feita a leitura de comandos e analisados usando o comando \textit{grep}

\begin{program}
  \begin{lstlisting}[language=bash]
file=/home/user/loginAttempts.txt
while read param
do
  echo -e "$param "
    if $param | grep -q "STRING"; then
      param = ${param#*'"'}; param=${param%'"'*}
      echo $param >> $file
      echo $(tail -10 $file) > $file
    fi
exit 0
\end{lstlisting}
\caption{/usr/local/etc/snmptrap/authenticationFailure.sh}
\end{program}

\section{Geração das Traps no Agente}

\par Agora que as \textit{traps} estão configuradas no Gerente, podemos configurar o agente para que as envie automáticamente.

\subsection{IF-MIB::linkDown}

\par Em computadores \textit{Linux} que utilizam o \textit{daemon NetworkManager} para gerenciar as interface de rede, já existe uma infraestrutura pronta para executar comandos no caso de mudanças no estado da conexão: o \textit{dispatcher.d}.

\par Todos os arquivos colocados na pasta \textit{/etc/NetworkManager/dispatcher.d} são executados sempre que há uma mudança em alguma interface de rede.
\par Para isso, foi criado o arquivo \textit{/etc/NetworkManager/dispatcher.d/99-snmptrap.sh}, que será a fonte da \textit{trap}.

\begin{program}
\begin{lstlisting}[language=bash]
interface=$1
event=$2
address=127.0.0.1
MIB=IF-MIB::linkDown
OID=linkDown

if [[ ( "$interface" == "enp2s0" && "$event" != "up" ) ]]
then
   snmptrap -v 2c -c public $address "" $MIB $OID s $interface
fi
exit 0

\end{lstlisting}
\caption{/etc/NetworkManager/dispatcher.d/99-snmptrap.sh}
\end{program}

\subsection{SNMPv2-MIB::authenticationFailure}
\par O serviço de autenticação do \textit{Linux} é chamado de PAM. Este serviço possui uma pasta de configurações, \textit{/etc/pam.d}, onde estão definidos os diferentes formatos de autenticação. Dentre eles, está o \texit{/etc/pam.d/sshd}, que permite as autenticações via SSH.

\par Para que seja executado um \textit{script} ao se executar autenticação via SSH, esse arquivo deve ser alterado, adicionando-se as seguintes linhas:

\begin{program}
\begin{myverb}
auth    [success=2 default=ignore]    pam_unix.so nullok_secure
auth    optional                      pam_exec.so /usr/local/etc/snmptrap/sshFailure.sh
\end{myverb}
\caption{/etc/pam.d/sshd}
\end{program}

\begin{program}
  \begin{lstlisting}[language=bash]
address=127.0.0.1
MIB=SNMPv2-MIB::authenticationFailure
OID=authenticationFailure
data=(date '+%Y-%m-%d %H:%M:%S')/$PAM_USER/$PAM_RHOST/InvalidPass
 
snmptrap -v 2c -c public $address "" $MIB $OID s $data
\end{lstlisting}
\caption{/etc/NetworkManager/dispatcher.d/99-snmptrap.sh}
\end{program}

\end{document}
