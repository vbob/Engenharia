%%%%%%%%%%%%%%%%%%%%%%%%%%%%%%%%%%%%%%%%%
%----------------------------------------------------------------------------------------
%	PACKAGES AND OTHER DOCUMENT CONFIGURATIONS
%----------------------------------------------------------------------------------------

\documentclass{article}

\input{structure.tex} % Include the file specifying the document structure and custom commands

%----------------------------------------------------------------------------------------
%	ASSIGNMENT INFORMATION
%----------------------------------------------------------------------------------------

\title{REDES II: Instalação do SNMP Tester} % Title of the assignment

\author{Vitor Bruno de Oliveira Barth\\ \texttt{vbob@vbob.com.br}} % Author name and email address

\date{Instituto Federal de Educação, Ciência e Tecnologia de Mato Grosso --- \today} % University, school and/or department name(s) and a date

%----------------------------------------------------------------------------------------

\begin{document}

\maketitle % Print the title

%----------------------------------------------------------------------------------------
%	INTRODUCTION
%----------------------------------------------------------------------------------------

\section{Introdução} % Unnumbered section

Foi desenvolvido o \textit{software} SNMP Tester. Este é baseado em Node.js com interface construída em Pug.js e Bootstrap.

\bigskip
\bigskip
\bigskip

\begin{figure}[!htb]
    \center{\includegraphics[width=\textwidth]
    {print.jpeg}}
    \caption{\label{fig:my-label} Tela do SNMP Tester}
\end{figure}


\pagebreak

%----------------------------------------------------------------------------------------
%	PROBLEM 1
%----------------------------------------------------------------------------------------

\section{Funcionalidades} % Numbered section

\par \par O SNMP Tester é capaz de realizar buscas SNMP de forma dinâmica, possuindo diversas configurações e um \textit{parser} personalizado.

\par É possível definir os seguintes parâmetros:

\begin{itemize}
	\item Endereço: URL ou Endereço IP do agente desejado
	\item Porta: Porta de Escuta do agente SNMP
	\item Comunidade: Comunidade SNMP
        \item Versão: Versão do SNMP (1 ou 2c)
        \item Método: SNMPGET para buscas de somente uma OID ou SNMPBULKWALK para explorar a árvore de MIBs
        \item Parâmetros adicionais: Operadores que podem ser concatenados ao final do comando SNMP
\end{itemize}

\par Ao final da requisição, as informações enviadas pelo agente são analisadas e exibidas em forma de tabela para facilitar a visualização e busca pelo usuário.

%----------------------------------------------------------------------------------------------------
%	PROBLEM 2
%----------------------------------------------------------------------------------------

\section{Instalação}

O SNMP Tester foi testado somente em ambientes Linux, podendo também ser compatível com outros Sistemas Operacionais.

\subsection{Pré-requisitos}

O \textit{parser} do SNMP Tester é construído sobre os executáveis do pacote \textit{net-snmp}. Por esta razão, é necessário que estes estejam instalados e incluídos na \$PATH.

Também é necessária a instalação do \textit{Node.js} em versão 8 ou superior.

\subsection{Download}

A versão mais atualizada está disponível no repositório http://github.com/vbob/snmpTester

\subsection{Instalação das Dependências}

Após realizar o \textit{download} do repositório e acessar o diretório criado, deve-se instalar as dependências com \textit{Node Package Manager} (npm).

% Command-line "screenshot"
\begin{commandline}
	\begin{verbatim}
		$ npm install
	\end{verbatim}
\end{commandline}

\subsection{Compilação}

\par Por ser construído em \textit{TypeScript}, é necessário que o código seja compilado. Para isso, executa-se o comando:

\begin{commandline}
\begin{verbatim}
$ npm run build
\end{verbatim}
\end{commandline}

%----------------------------------------------------------------------------------------

\section{Utilização}

Após instalado, será possível iniciar o servidor através do comando

\begin{commandline}
\begin{verbatim}
$ npm start
\end{verbatim}
\end{commandline}

O servidor estará disponível para acesso no endereço http://127.0.0.1:8080



\end{document}
